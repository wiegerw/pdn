%% Generated by Sphinx.
\def\sphinxdocclass{report}
\documentclass[letterpaper,10pt,english]{sphinxmanual}
\ifdefined\pdfpxdimen
   \let\sphinxpxdimen\pdfpxdimen\else\newdimen\sphinxpxdimen
\fi \sphinxpxdimen=.75bp\relax
\ifdefined\pdfimageresolution
    \pdfimageresolution= \numexpr \dimexpr1in\relax/\sphinxpxdimen\relax
\fi
%% let collapsible pdf bookmarks panel have high depth per default
\PassOptionsToPackage{bookmarksdepth=5}{hyperref}

\PassOptionsToPackage{booktabs}{sphinx}
\PassOptionsToPackage{colorrows}{sphinx}

\PassOptionsToPackage{warn}{textcomp}
\usepackage[utf8]{inputenc}
\ifdefined\DeclareUnicodeCharacter
% support both utf8 and utf8x syntaxes
  \ifdefined\DeclareUnicodeCharacterAsOptional
    \def\sphinxDUC#1{\DeclareUnicodeCharacter{"#1}}
  \else
    \let\sphinxDUC\DeclareUnicodeCharacter
  \fi
  \sphinxDUC{00A0}{\nobreakspace}
  \sphinxDUC{2500}{\sphinxunichar{2500}}
  \sphinxDUC{2502}{\sphinxunichar{2502}}
  \sphinxDUC{2514}{\sphinxunichar{2514}}
  \sphinxDUC{251C}{\sphinxunichar{251C}}
  \sphinxDUC{2572}{\textbackslash}
\fi
\usepackage{cmap}
\usepackage[T1]{fontenc}
\usepackage{amsmath,amssymb,amstext}
\usepackage{babel}



\usepackage{tgtermes}
\usepackage{tgheros}
\renewcommand{\ttdefault}{txtt}



\usepackage[Bjarne]{fncychap}
\usepackage{sphinx}

\fvset{fontsize=auto}
\usepackage{geometry}


% Include hyperref last.
\usepackage{hyperref}
% Fix anchor placement for figures with captions.
\usepackage{hypcap}% it must be loaded after hyperref.
% Set up styles of URL: it should be placed after hyperref.
\urlstyle{same}


\usepackage{sphinxmessages}
\setcounter{tocdepth}{1}



\title{PDN 3.0 Standard}
\date{Jan 09, 2025}
\release{3.0}
\author{Wieger Wesselink}
\newcommand{\sphinxlogo}{\vbox{}}
\renewcommand{\releasename}{Release}
\makeindex
\begin{document}

\ifdefined\shorthandoff
  \ifnum\catcode`\=\string=\active\shorthandoff{=}\fi
  \ifnum\catcode`\"=\active\shorthandoff{"}\fi
\fi

\pagestyle{empty}
\sphinxmaketitle
\pagestyle{plain}
\sphinxtableofcontents
\pagestyle{normal}
\phantomsection\label{\detokenize{index::doc}}


\sphinxstepscope


\chapter{Introduction}
\label{\detokenize{introduction:introduction}}\label{\detokenize{introduction::doc}}
\sphinxAtStartPar
This document defines the official portable draughts notation standard, PDN 3.0.
The goal of this document is to give a formal definition of PDN.
Nowadays there are many conflicts between draughts programs regarding their
interpretation of valid PDN. This document aims to take away the sources of
conflicts, by defining grammars, and by giving additional restrictions. Note
that the PDN definition as given in this document slightly deviates from earlier
versions. This has been done to make parsing PDN easier. Furthermore, some
extensions are defined to support setting up positions and to add clock times.

\sphinxAtStartPar
In \sphinxcite{introduction:wikipedia}, \sphinxcite{introduction:sage}, \sphinxcite{introduction:nemesis} and \sphinxcite{introduction:grimminck} earlier versions
of PDN definitions can be found. PDN was derived from the PGN standard
in chess, see \sphinxcite{introduction:pgn}. The PDN proposal in this document takes the PDN 2.0 draft
\sphinxcite{introduction:nemesis} as a starting point, since it is more complete and precise than the
earlier ones. Several additions, restrictions and extensions to the PDN 2.0 draft
are made.

\sphinxAtStartPar
This document first discusses PDN tags, then a grammar for PDN 3.0 is given, followed
by some PDN extensions, and some example implementations of the grammar. The
design of the grammar was influenced by a number of issues of the existing
PDN standards. These issues are discussed in the {\hyperref[\detokenize{issues:issues-section}]{\sphinxcrossref{\DUrole{std}{\DUrole{std-ref}{PDN parsing issues}}}}} section.

\sphinxAtStartPar
For convenience a Java Webstart \sphinxhref{http://10x10.org/pdn/test/index.html}{PDN 3.0 Checker}
is available, that can be used for checking the validity of a PDN file.


\section{PDN Example}
\label{\detokenize{introduction:pdn-example}}
\sphinxAtStartPar
A PDN file consists of a section of tags (key\sphinxhyphen{}value pairs between brackets)
followed by a section with moves and variations. A typical example is:

\begin{sphinxVerbatim}[commandchars=\\\{\}]
\PYG{p}{[}\PYG{n}{Event} \PYG{l+s+s2}{\PYGZdq{}}\PYG{l+s+s2}{FMJD World Championship}\PYG{l+s+s2}{\PYGZdq{}}\PYG{p}{]}
\PYG{p}{[}\PYG{n}{Site} \PYG{l+s+s2}{\PYGZdq{}}\PYG{l+s+s2}{Hardenberg, NED}\PYG{l+s+s2}{\PYGZdq{}}\PYG{p}{]}
\PYG{p}{[}\PYG{n}{Date} \PYG{l+s+s2}{\PYGZdq{}}\PYG{l+s+s2}{2007.05.19}\PYG{l+s+s2}{\PYGZdq{}}\PYG{p}{]}
\PYG{p}{[}\PYG{n}{Round} \PYG{l+s+s2}{\PYGZdq{}}\PYG{l+s+s2}{7}\PYG{l+s+s2}{\PYGZdq{}}\PYG{p}{]}
\PYG{p}{[}\PYG{n}{White} \PYG{l+s+s2}{\PYGZdq{}}\PYG{l+s+s2}{Mikhalchenko,I.}\PYG{l+s+s2}{\PYGZdq{}}\PYG{p}{]}
\PYG{p}{[}\PYG{n}{Black} \PYG{l+s+s2}{\PYGZdq{}}\PYG{l+s+s2}{Ndjofang,J.}\PYG{l+s+s2}{\PYGZdq{}}\PYG{p}{]}
\PYG{p}{[}\PYG{n}{Result} \PYG{l+s+s2}{\PYGZdq{}}\PYG{l+s+s2}{0\PYGZhy{}2}\PYG{l+s+s2}{\PYGZdq{}}\PYG{p}{]}
\PYG{p}{[}\PYG{n}{GameType} \PYG{l+s+s2}{\PYGZdq{}}\PYG{l+s+s2}{20}\PYG{l+s+s2}{\PYGZdq{}}\PYG{p}{]}
\PYG{p}{[}\PYG{n}{WhiteTime} \PYG{l+s+s2}{\PYGZdq{}}\PYG{l+s+s2}{1:36}\PYG{l+s+s2}{\PYGZdq{}}\PYG{p}{]}
\PYG{p}{[}\PYG{n}{BlackTime} \PYG{l+s+s2}{\PYGZdq{}}\PYG{l+s+s2}{1:17}\PYG{l+s+s2}{\PYGZdq{}}\PYG{p}{]}

 \PYG{l+m+mf}{1.32}\PYG{o}{\PYGZhy{}}\PYG{l+m+mi}{28} \PYG{l+m+mi}{17}\PYG{o}{\PYGZhy{}}\PYG{l+m+mi}{22}  \PYG{l+m+mf}{2.28}\PYG{n}{x17} \PYG{l+m+mi}{11}\PYG{n}{x22}  \PYG{l+m+mf}{3.37}\PYG{o}{\PYGZhy{}}\PYG{l+m+mi}{32}  \PYG{l+m+mi}{6}\PYG{o}{\PYGZhy{}}\PYG{l+m+mi}{11}  \PYG{l+m+mf}{4.41}\PYG{o}{\PYGZhy{}}\PYG{l+m+mi}{37} \PYG{l+m+mi}{12}\PYG{o}{\PYGZhy{}}\PYG{l+m+mi}{17}  \PYG{l+m+mf}{5.46}\PYG{o}{\PYGZhy{}}\PYG{l+m+mi}{41}  \PYG{l+m+mi}{8}\PYG{o}{\PYGZhy{}}\PYG{l+m+mi}{12}
 \PYG{l+m+mf}{6.34}\PYG{o}{\PYGZhy{}}\PYG{l+m+mi}{30}  \PYG{l+m+mi}{2}\PYG{o}{\PYGZhy{}}\PYG{l+m+mi}{8}   \PYG{l+m+mf}{7.30}\PYG{o}{\PYGZhy{}}\PYG{l+m+mi}{25} \PYG{l+m+mi}{19}\PYG{o}{\PYGZhy{}}\PYG{l+m+mi}{23}  \PYG{l+m+mf}{8.35}\PYG{o}{\PYGZhy{}}\PYG{l+m+mi}{30}  \PYG{l+m+mi}{1}\PYG{o}{\PYGZhy{}}\PYG{l+m+mi}{6}   \PYG{l+m+mf}{9.40}\PYG{o}{\PYGZhy{}}\PYG{l+m+mi}{35} \PYG{l+m+mi}{13}\PYG{o}{\PYGZhy{}}\PYG{l+m+mi}{19} \PYG{l+m+mf}{10.31}\PYG{o}{\PYGZhy{}}\PYG{l+m+mi}{27} \PYG{l+m+mi}{22}\PYG{n}{x31}
\PYG{l+m+mf}{11.36}\PYG{n}{x27}  \PYG{l+m+mi}{9}\PYG{o}{\PYGZhy{}}\PYG{l+m+mi}{13} \PYG{l+m+mf}{12.33}\PYG{o}{\PYGZhy{}}\PYG{l+m+mi}{28}  \PYG{l+m+mi}{4}\PYG{o}{\PYGZhy{}}\PYG{l+m+mi}{9}  \PYG{l+m+mf}{13.41}\PYG{o}{\PYGZhy{}}\PYG{l+m+mi}{36} \PYG{l+m+mi}{17}\PYG{o}{\PYGZhy{}}\PYG{l+m+mi}{22} \PYG{l+m+mf}{14.28}\PYG{n}{x17} \PYG{l+m+mi}{11}\PYG{n}{x31} \PYG{l+m+mf}{15.37}\PYG{n}{x26} \PYG{l+m+mi}{23}\PYG{o}{\PYGZhy{}}\PYG{l+m+mi}{28}
\PYG{l+m+mf}{16.32}\PYG{n}{x23} \PYG{l+m+mi}{19}\PYG{n}{x28} \PYG{l+m+mf}{17.42}\PYG{o}{\PYGZhy{}}\PYG{l+m+mi}{37} \PYG{l+m+mi}{20}\PYG{o}{\PYGZhy{}}\PYG{l+m+mi}{24} \PYG{l+m+mf}{18.30}\PYG{n}{x19} \PYG{l+m+mi}{14}\PYG{n}{x23} \PYG{l+m+mf}{19.37}\PYG{o}{\PYGZhy{}}\PYG{l+m+mi}{31} \PYG{l+m+mi}{16}\PYG{o}{\PYGZhy{}}\PYG{l+m+mi}{21} \PYG{l+m+mf}{20.26}\PYG{n}{x17} \PYG{l+m+mi}{12}\PYG{n}{x21}
\PYG{l+m+mf}{21.31}\PYG{o}{\PYGZhy{}}\PYG{l+m+mi}{27} \PYG{l+m+mi}{21}\PYG{n}{x32} \PYG{l+m+mf}{22.38}\PYG{n}{x27}  \PYG{l+m+mi}{6}\PYG{o}{\PYGZhy{}}\PYG{l+m+mi}{11} \PYG{l+m+mf}{23.47}\PYG{o}{\PYGZhy{}}\PYG{l+m+mi}{42} \PYG{l+m+mi}{15}\PYG{o}{\PYGZhy{}}\PYG{l+m+mi}{20} \PYG{l+m+mf}{24.25}\PYG{n}{x14} \PYG{l+m+mi}{10}\PYG{n}{x19} \PYG{l+m+mf}{25.39}\PYG{o}{\PYGZhy{}}\PYG{l+m+mi}{33} \PYG{l+m+mi}{28}\PYG{n}{x39}
\PYG{l+m+mf}{26.44}\PYG{n}{x33}  \PYG{l+m+mi}{8}\PYG{o}{\PYGZhy{}}\PYG{l+m+mi}{12} \PYG{l+m+mf}{27.42}\PYG{o}{\PYGZhy{}}\PYG{l+m+mi}{38} \PYG{l+m+mi}{23}\PYG{o}{\PYGZhy{}}\PYG{l+m+mi}{28} \PYG{l+m+mf}{28.33}\PYG{n}{x22} \PYG{l+m+mi}{12}\PYG{o}{\PYGZhy{}}\PYG{l+m+mi}{17} \PYG{l+m+mf}{29.49}\PYG{o}{\PYGZhy{}}\PYG{l+m+mi}{44} \PYG{l+m+mi}{17}\PYG{n}{x28} \PYG{l+m+mf}{30.38}\PYG{o}{\PYGZhy{}}\PYG{l+m+mi}{33} \PYG{l+m+mi}{28}\PYG{n}{x39}
\PYG{l+m+mf}{31.44}\PYG{n}{x33} \PYG{l+m+mi}{18}\PYG{o}{\PYGZhy{}}\PYG{l+m+mi}{22} \PYG{l+m+mf}{32.27}\PYG{n}{x18} \PYG{l+m+mi}{13}\PYG{n}{x22} \PYG{l+m+mf}{33.43}\PYG{o}{\PYGZhy{}}\PYG{l+m+mi}{38} \PYG{l+m+mi}{19}\PYG{o}{\PYGZhy{}}\PYG{l+m+mi}{23} \PYG{l+m+mf}{34.38}\PYG{o}{\PYGZhy{}}\PYG{l+m+mi}{32} \PYG{l+m+mi}{11}\PYG{o}{\PYGZhy{}}\PYG{l+m+mi}{17} \PYG{l+m+mf}{35.32}\PYG{o}{\PYGZhy{}}\PYG{l+m+mi}{27} \PYG{l+m+mi}{22}\PYG{n}{x31}
\PYG{l+m+mf}{36.36}\PYG{n}{x27}  \PYG{l+m+mi}{9}\PYG{o}{\PYGZhy{}}\PYG{l+m+mi}{13} \PYG{l+m+mf}{37.45}\PYG{o}{\PYGZhy{}}\PYG{l+m+mi}{40} \PYG{l+m+mi}{13}\PYG{o}{\PYGZhy{}}\PYG{l+m+mi}{18} \PYG{o}{*}
\end{sphinxVerbatim}


\section{Reading and writing}
\label{\detokenize{introduction:reading-and-writing}}
\sphinxAtStartPar
Programmers are encouraged to follow the PDN 3.0 standard closely when writing
PDN. When reading, it is recommended to treat the input much more liberally.
To this end an example of a liberal reading grammar is given. Also the character
encoding requirement can be relaxed.


\section{Acknowledgements}
\label{\detokenize{introduction:acknowledgements}}
\sphinxAtStartPar
Thanks to everyone who took part in the discussions on the World Draughts Forum \sphinxcite{introduction:forum},
and to Rein Halbersma for leading the review.


\section{References}
\label{\detokenize{introduction:references}}
\sphinxstepscope


\chapter{Character encoding}
\label{\detokenize{encoding:character-encoding}}\label{\detokenize{encoding::doc}}
\sphinxAtStartPar
For PDN 3.0 documents the UTF\sphinxhyphen{}8 character encoding is required. Note that this includes ASCII encoded documents.
When reading PDN, it is recommended to accept other encodings too, like ISO 8859/1 (Latin 1).

\sphinxAtStartPar
Note that PDN differs from PGN in this respect, see also \sphinxurl{http://www.saremba.de/chessgml/standards/pgn/pgn-complete.htm\#c4.1}.
The reason for this choice is that UTF\sphinxhyphen{}8 supports many different character sets like Cyrillic, Chinese, Japanese etc.
UTF\sphinxhyphen{}8 is also used in the XHTML standard.

\sphinxAtStartPar
There is no maximum line length defined for a PDN document. So it is allowed to put an entire game on one line.

\sphinxstepscope


\chapter{PDN Tags}
\label{\detokenize{pdntags:pdn-tags}}\label{\detokenize{pdntags::doc}}
\sphinxAtStartPar
This section gives an overview of predefined PDN tags. Tags are key\sphinxhyphen{}value pairs
between brackets, in which the value is surrounded by double quotes. More extensive
documentation about most of the tags given here can be found in \sphinxcite{introduction:pgn}. The number
of predefined tags has been kept limited. It is allowed to add user defined tags.
Note that a key must start with a capital.

\sphinxAtStartPar
We adopt the rule that a tag is omitted if it has no value. This deviates from \sphinxcite{introduction:pgn},
that defines an explicit \sphinxstyleemphasis{no value} for some tags in the form of a minus sign \sphinxcode{\sphinxupquote{"\sphinxhyphen{}"}}.
To denote that a tag has an unknown value, in most cases a \sphinxcode{\sphinxupquote{"?"}} value can be used. For
example \sphinxcode{\sphinxupquote{{[}Event "?"{]}}} denotes that the event of the game was unknown. Some tags have
a more specific unknown value. Instead of a \sphinxcode{\sphinxupquote{"?"}} value, it is also allowed to
use the empty string \sphinxcode{\sphinxupquote{""}}.

\sphinxAtStartPar
This section gives an overview of the predefined tags, followed by some examples
and additional details about the tags.


\section{‘Mandatory’ tags}
\label{\detokenize{pdntags:mandatory-tags}}

\begin{savenotes}\sphinxattablestart
\sphinxthistablewithglobalstyle
\centering
\begin{tabulary}{\linewidth}[t]{TTT}
\sphinxtoprule
\sphinxstyletheadfamily 
\sphinxAtStartPar
Tag
&\sphinxstyletheadfamily 
\sphinxAtStartPar
Description
&\sphinxstyletheadfamily 
\sphinxAtStartPar
Unknown value
\\
\sphinxmidrule
\sphinxtableatstartofbodyhook
\sphinxAtStartPar
Event
&
\sphinxAtStartPar
Name of the tournament or match event, see \sphinxcite{introduction:nemesis}
&
\sphinxAtStartPar
“?”
\\
\sphinxhline
\sphinxAtStartPar
Site
&
\sphinxAtStartPar
Location of the event, see \sphinxcite{introduction:nemesis}
&
\sphinxAtStartPar
“?”
\\
\sphinxhline
\sphinxAtStartPar
Date
&
\sphinxAtStartPar
Starting date of the game, see \sphinxcite{introduction:nemesis}
&
\sphinxAtStartPar
“????.??.??”
\\
\sphinxhline
\sphinxAtStartPar
Round
&
\sphinxAtStartPar
Playing round ordinal of the game, see \sphinxcite{introduction:nemesis}
&
\sphinxAtStartPar
“?”
\\
\sphinxhline
\sphinxAtStartPar
White
&
\sphinxAtStartPar
Player of the White pieces, see \sphinxcite{introduction:nemesis}
&
\sphinxAtStartPar
“?”
\\
\sphinxhline
\sphinxAtStartPar
Black
&
\sphinxAtStartPar
Player of the Black pieces, see \sphinxcite{introduction:nemesis}
&
\sphinxAtStartPar
“?”
\\
\sphinxhline
\sphinxAtStartPar
Result
&
\sphinxAtStartPar
Result of the game, see \sphinxcite{introduction:nemesis}
&
\sphinxAtStartPar
“*”
\\
\sphinxbottomrule
\end{tabulary}
\sphinxtableafterendhook\par
\sphinxattableend\end{savenotes}

\sphinxAtStartPar
N.B. The name \sphinxstyleemphasis{mandatory tags} is misleading, since these tags are not mandatory
for an arbitrary PDN file. It is however recommended to include all of the mandatory
tags in tournament games.


\section{Result tag}
\label{\detokenize{pdntags:result-tag}}\begin{quote}

\sphinxAtStartPar
The \sphinxcode{\sphinxupquote{Result}} tag is used to specify whether a game ended in a win, draw or a loss.
Each game type has a specific set of allowed values for the \sphinxcode{\sphinxupquote{Result}} tag.


\begin{savenotes}\sphinxattablestart
\sphinxthistablewithglobalstyle
\centering
\begin{tabulary}{\linewidth}[t]{TT}
\sphinxtoprule
\sphinxstyletheadfamily 
\sphinxAtStartPar
ResultType
&\sphinxstyletheadfamily 
\sphinxAtStartPar
Allowed result values
\\
\sphinxmidrule
\sphinxtableatstartofbodyhook
\sphinxAtStartPar
Default
&
\sphinxAtStartPar
1\sphinxhyphen{}0, 0\sphinxhyphen{}1, 1/2\sphinxhyphen{}1/2, 0\sphinxhyphen{}0, *
\\
\sphinxhline
\sphinxAtStartPar
International
&
\sphinxAtStartPar
2\sphinxhyphen{}0, 0\sphinxhyphen{}2, 1\sphinxhyphen{}1, 0\sphinxhyphen{}0, *
\\
\sphinxbottomrule
\end{tabulary}
\sphinxtableafterendhook\par
\sphinxattableend\end{savenotes}

\sphinxAtStartPar
In the section {\hyperref[\detokenize{gametype:gametype-section}]{\sphinxcrossref{\DUrole{std}{\DUrole{std-ref}{GameType tag}}}}} it is specified which result values belong
to which game type.

\sphinxAtStartPar
The value \sphinxcode{\sphinxupquote{*}} denotes that the game was unfinished, or there is no result available.
The value \sphinxcode{\sphinxupquote{0\sphinxhyphen{}0}} denotes that the game was declared lost for both players.

\sphinxAtStartPar
When reading a PDN document, it is recommended to accept arbitrary strings as results.
Note that when the \sphinxcode{\sphinxupquote{ResultFormat}} tag is set, the set of allowed values for the \sphinxcode{\sphinxupquote{Result}}
tag is overruled.

\sphinxAtStartPar
Examples:

\begin{sphinxVerbatim}[commandchars=\\\{\}]
\PYG{p}{[}\PYG{n}{Result} \PYG{l+s+s2}{\PYGZdq{}}\PYG{l+s+s2}{1/2\PYGZhy{}1/2}\PYG{l+s+s2}{\PYGZdq{}}\PYG{p}{]}
\end{sphinxVerbatim}

\sphinxAtStartPar
Sometimes tournaments are being played with different result values. To this end the
\sphinxcode{\sphinxupquote{ResultFormat}} tag is defined. Below a table is given for some common result formats:


\begin{savenotes}\sphinxattablestart
\sphinxthistablewithglobalstyle
\centering
\begin{tabulary}{\linewidth}[t]{TT}
\sphinxtoprule
\sphinxstyletheadfamily 
\sphinxAtStartPar
ResultFormat
&\sphinxstyletheadfamily 
\sphinxAtStartPar
Allowed result values
\\
\sphinxmidrule
\sphinxtableatstartofbodyhook
\sphinxAtStartPar
Plus draw
&
\sphinxAtStartPar
2\sphinxhyphen{}0, 0\sphinxhyphen{}2, 1\sphinxhyphen{}1, 0\sphinxhyphen{}0, 1+ \sphinxhyphen{} 1\sphinxhyphen{}, 1\sphinxhyphen{} \sphinxhyphen{} 1+, *
\\
\sphinxhline
\sphinxAtStartPar
Delfts
&
\sphinxAtStartPar
2\sphinxhyphen{}0, 0\sphinxhyphen{}2, 1 1/2 \sphinxhyphen{} 1/2, 1/2 \sphinxhyphen{} 1 1/2, 1\sphinxhyphen{}1, 0\sphinxhyphen{}0, *
\\
\sphinxhline
\sphinxAtStartPar
Goes
&
\sphinxAtStartPar
2\sphinxhyphen{}0, 0\sphinxhyphen{}2, …, \sphinxhyphen{}0.98\sphinxhyphen{}1.02, \sphinxhyphen{}0.99\sphinxhyphen{}1.01, 1\sphinxhyphen{}1, 1.01\sphinxhyphen{}0.99, 1.02\sphinxhyphen{}0.98, …, 0\sphinxhyphen{}0, *
\\
\sphinxbottomrule
\end{tabulary}
\sphinxtableafterendhook\par
\sphinxattableend\end{savenotes}

\sphinxAtStartPar
When the \sphinxcode{\sphinxupquote{ResultFormat}} tag is set, the \sphinxcode{\sphinxupquote{Result}} tag must have a corresponding allowed value.

\sphinxAtStartPar
Examples:

\begin{sphinxVerbatim}[commandchars=\\\{\}]
\PYG{p}{[}\PYG{n}{ResultFormat} \PYG{l+s+s2}{\PYGZdq{}}\PYG{l+s+s2}{Plus draw}\PYG{l+s+s2}{\PYGZdq{}}\PYG{p}{]}
\PYG{p}{[}\PYG{n}{Result} \PYG{l+s+s2}{\PYGZdq{}}\PYG{l+s+s2}{1+ \PYGZhy{} 1\PYGZhy{}}\PYG{l+s+s2}{\PYGZdq{}}\PYG{p}{]}
\end{sphinxVerbatim}
\end{quote}


\section{Player related tags}
\label{\detokenize{pdntags:player-related-tags}}

\begin{savenotes}\sphinxattablestart
\sphinxthistablewithglobalstyle
\centering
\begin{tabulary}{\linewidth}[t]{TT}
\sphinxtoprule
\sphinxstyletheadfamily 
\sphinxAtStartPar
Tag
&\sphinxstyletheadfamily 
\sphinxAtStartPar
Description
\\
\sphinxmidrule
\sphinxtableatstartofbodyhook
\sphinxAtStartPar
WhiteTitle, BlackTitle
&
\sphinxAtStartPar
FMJD titles of the players
\\
\sphinxhline
\sphinxAtStartPar
WhiteRating, BlackRating
&
\sphinxAtStartPar
FMJD rating
\\
\sphinxhline
\sphinxAtStartPar
WhiteNA, BlackNA
&
\sphinxAtStartPar
E\sphinxhyphen{}mail or network addresses of the players
\\
\sphinxhline
\sphinxAtStartPar
WhiteType, BlackType
&
\sphinxAtStartPar
Player types (“human” or “program”)
\\
\sphinxbottomrule
\end{tabulary}
\sphinxtableafterendhook\par
\sphinxattableend\end{savenotes}

\sphinxAtStartPar
The tags \sphinxcode{\sphinxupquote{WhiteRating}} and \sphinxcode{\sphinxupquote{BlackRating}} are named WhiteElo and BlackElo in chess.


\section{Event related tags}
\label{\detokenize{pdntags:event-related-tags}}

\begin{savenotes}\sphinxattablestart
\sphinxthistablewithglobalstyle
\centering
\begin{tabulary}{\linewidth}[t]{TT}
\sphinxtoprule
\sphinxstyletheadfamily 
\sphinxAtStartPar
Tag
&\sphinxstyletheadfamily 
\sphinxAtStartPar
Description
\\
\sphinxmidrule
\sphinxtableatstartofbodyhook
\sphinxAtStartPar
EventDate
&
\sphinxAtStartPar
Starting date of the event
\\
\sphinxhline
\sphinxAtStartPar
EventSponsor
&
\sphinxAtStartPar
Sponsor of the event
\\
\sphinxhline
\sphinxAtStartPar
Section
&
\sphinxAtStartPar
Playing section of a tournament (e.g., “Open” or “Reserve”)
\\
\sphinxhline
\sphinxAtStartPar
Stage
&
\sphinxAtStartPar
Stage of a multistage event (e.g., “Preliminary” or “Semifinal”)
\\
\sphinxhline
\sphinxAtStartPar
Board
&
\sphinxAtStartPar
Board number in a team event or in a simultaneous exhibition
\\
\sphinxbottomrule
\end{tabulary}
\sphinxtableafterendhook\par
\sphinxattableend\end{savenotes}


\section{Game related tags}
\label{\detokenize{pdntags:game-related-tags}}

\begin{savenotes}\sphinxattablestart
\sphinxthistablewithglobalstyle
\centering
\begin{tabulary}{\linewidth}[t]{TT}
\sphinxtoprule
\sphinxstyletheadfamily 
\sphinxAtStartPar
Tag
&\sphinxstyletheadfamily 
\sphinxAtStartPar
Description
\\
\sphinxmidrule
\sphinxtableatstartofbodyhook
\sphinxAtStartPar
GameType
&
\sphinxAtStartPar
Type of the game, see {\hyperref[\detokenize{gametype:gametype-section}]{\sphinxcrossref{\DUrole{std}{\DUrole{std-ref}{GameType tag}}}}}
\\
\sphinxhline
\sphinxAtStartPar
FEN
&
\sphinxAtStartPar
The position at the start of the game, see {\hyperref[\detokenize{fen:fen-section}]{\sphinxcrossref{\DUrole{std}{\DUrole{std-ref}{FEN tag}}}}}
\\
\sphinxhline
\sphinxAtStartPar
PlyCount
&
\sphinxAtStartPar
The number of ply (moves) in the game
\\
\sphinxhline
\sphinxAtStartPar
Termination
&
\sphinxAtStartPar
Describes the reason for conclusion of the game, see \sphinxcite{introduction:pgn}
\\
\sphinxbottomrule
\end{tabulary}
\sphinxtableafterendhook\par
\sphinxattableend\end{savenotes}

\sphinxAtStartPar
The \sphinxcode{\sphinxupquote{GameType}} tag is specific for draughts, and is used to distinguish between
the different draughts variants.


\section{Clock related tags}
\label{\detokenize{pdntags:clock-related-tags}}

\begin{savenotes}\sphinxattablestart
\sphinxthistablewithglobalstyle
\centering
\begin{tabulary}{\linewidth}[t]{TT}
\sphinxtoprule
\sphinxstyletheadfamily 
\sphinxAtStartPar
Tag
&\sphinxstyletheadfamily 
\sphinxAtStartPar
Description
\\
\sphinxmidrule
\sphinxtableatstartofbodyhook
\sphinxAtStartPar
TimeControl
&
\sphinxAtStartPar
Time control settings for both players, see \sphinxcite{introduction:pgn}
\\
\sphinxhline
\sphinxAtStartPar
TimeControlWhite
&
\sphinxAtStartPar
Time control settings for the white player
\\
\sphinxhline
\sphinxAtStartPar
TimeControlBlack
&
\sphinxAtStartPar
Time control settings for the black player
\\
\sphinxhline
\sphinxAtStartPar
WhiteTime
&
\sphinxAtStartPar
Time used by the White player at the end of the game
\\
\sphinxhline
\sphinxAtStartPar
BlackTime
&
\sphinxAtStartPar
Time used by the Black player at the end of the game
\\
\sphinxbottomrule
\end{tabulary}
\sphinxtableafterendhook\par
\sphinxattableend\end{savenotes}

\sphinxAtStartPar
The \sphinxcode{\sphinxupquote{WhiteTime}} and \sphinxcode{\sphinxupquote{BlackTime}} tags are new. It is common practice
to record the time used by both players, so it seems useful to define a
tag for it. The \sphinxcode{\sphinxupquote{TimeControlWhite}} and \sphinxcode{\sphinxupquote{TimeControlBlack}} tags can
be used when the players start with different times on the clock.
This is for example the case in Georgiev\sphinxhyphen{}Lehmann tie\sphinxhyphen{}breaks.


\section{Time and date tags}
\label{\detokenize{pdntags:time-and-date-tags}}

\begin{savenotes}\sphinxattablestart
\sphinxthistablewithglobalstyle
\centering
\begin{tabulary}{\linewidth}[t]{TT}
\sphinxtoprule
\sphinxstyletheadfamily 
\sphinxAtStartPar
Tag
&\sphinxstyletheadfamily 
\sphinxAtStartPar
Description
\\
\sphinxmidrule
\sphinxtableatstartofbodyhook
\sphinxAtStartPar
Time
&
\sphinxAtStartPar
Time\sphinxhyphen{}of\sphinxhyphen{}day value in “HH:MM:SS” format
\\
\sphinxhline
\sphinxAtStartPar
UTCTime
&
\sphinxAtStartPar
Time\sphinxhyphen{}of\sphinxhyphen{}day in Universal Coordinated Time format
\\
\sphinxhline
\sphinxAtStartPar
UTCDate
&
\sphinxAtStartPar
Date in Universal Coordinated Time format
\\
\sphinxbottomrule
\end{tabulary}
\sphinxtableafterendhook\par
\sphinxattableend\end{savenotes}


\section{Miscellaneous tags}
\label{\detokenize{pdntags:miscellaneous-tags}}

\begin{savenotes}\sphinxattablestart
\sphinxthistablewithglobalstyle
\centering
\begin{tabulary}{\linewidth}[t]{TT}
\sphinxtoprule
\sphinxstyletheadfamily 
\sphinxAtStartPar
Tag
&\sphinxstyletheadfamily 
\sphinxAtStartPar
Description
\\
\sphinxmidrule
\sphinxtableatstartofbodyhook
\sphinxAtStartPar
Annotator
&
\sphinxAtStartPar
Identifies the annotator or annotators of the game
\\
\sphinxbottomrule
\end{tabulary}
\sphinxtableafterendhook\par
\sphinxattableend\end{savenotes}


\section{Problemism related tags}
\label{\detokenize{pdntags:problemism-related-tags}}

\begin{savenotes}\sphinxattablestart
\sphinxthistablewithglobalstyle
\centering
\begin{tabulary}{\linewidth}[t]{TT}
\sphinxtoprule
\sphinxstyletheadfamily 
\sphinxAtStartPar
Tag
&\sphinxstyletheadfamily 
\sphinxAtStartPar
Description
\\
\sphinxmidrule
\sphinxtableatstartofbodyhook
\sphinxAtStartPar
Author
&
\sphinxAtStartPar
Author(s) of the analysis or composition
\\
\sphinxhline
\sphinxAtStartPar
Publication
&
\sphinxAtStartPar
Original publication of the analysis or composition
\\
\sphinxhline
\sphinxAtStartPar
PublicationDate
&
\sphinxAtStartPar
Date of the original publication
\\
\sphinxbottomrule
\end{tabulary}
\sphinxtableafterendhook\par
\sphinxattableend\end{savenotes}

\sphinxAtStartPar
These tags are new. PDN can be used to store databases with problems,
so it seems useful to define tags to support this.


\subsection{Details and Examples}
\label{\detokenize{pdntags:details-and-examples}}\begin{description}
\sphinxlineitem{\sphinxstylestrong{Event Tag}}
\sphinxAtStartPar
The Event tag specifies the event. Abbreviations are to be avoided.

\sphinxAtStartPar
Example:

\begin{sphinxVerbatim}[commandchars=\\\{\}]
\PYG{p}{[}\PYG{n}{Event} \PYG{l+s+s2}{\PYGZdq{}}\PYG{l+s+s2}{FMJD World Championship}\PYG{l+s+s2}{\PYGZdq{}}\PYG{p}{]}
\end{sphinxVerbatim}

\sphinxlineitem{\sphinxstylestrong{Site Tag:}}
\sphinxAtStartPar
The Site tag specifies the location. Use IOC country codes to denote
countries.

\sphinxAtStartPar
Examples:

\begin{sphinxVerbatim}[commandchars=\\\{\}]
\PYG{p}{[}\PYG{n}{Site} \PYG{l+s+s2}{\PYGZdq{}}\PYG{l+s+s2}{New York City, NY USA}\PYG{l+s+s2}{\PYGZdq{}}\PYG{p}{]}
\PYG{p}{[}\PYG{n}{Site} \PYG{l+s+s2}{\PYGZdq{}}\PYG{l+s+s2}{St. Petersburg RUS}\PYG{l+s+s2}{\PYGZdq{}}\PYG{p}{]}
\PYG{p}{[}\PYG{n}{Site} \PYG{l+s+s2}{\PYGZdq{}}\PYG{l+s+s2}{Riga LAT}\PYG{l+s+s2}{\PYGZdq{}}\PYG{p}{]}
\end{sphinxVerbatim}

\sphinxlineitem{\sphinxstylestrong{Date Tag}}
\sphinxAtStartPar
The Date tag must be specified in \sphinxcode{\sphinxupquote{YYYY.MM.DD}} format. Question
marks may be used for unknown fields.

\sphinxAtStartPar
A regular expression for Date values is:

\begin{sphinxVerbatim}[commandchars=\\\{\}]
([0\PYGZhy{}9]\PYGZob{}4\PYGZcb{}|[?]\PYGZob{}4\PYGZcb{})\PYGZbs{}.([0\PYGZhy{}9]\PYGZob{}2\PYGZcb{}|[?]\PYGZob{}2\PYGZcb{})\PYGZbs{}.([0\PYGZhy{}9]\PYGZob{}2\PYGZcb{}|[?]\PYGZob{}2\PYGZcb{})
\end{sphinxVerbatim}

\sphinxAtStartPar
Examples:

\begin{sphinxVerbatim}[commandchars=\\\{\}]
\PYG{p}{[}\PYG{n}{Date} \PYG{l+s+s2}{\PYGZdq{}}\PYG{l+s+s2}{1996.12.28}\PYG{l+s+s2}{\PYGZdq{}}\PYG{p}{]}
\PYG{p}{[}\PYG{n}{Date} \PYG{l+s+s2}{\PYGZdq{}}\PYG{l+s+s2}{2007.??.??}\PYG{l+s+s2}{\PYGZdq{}}\PYG{p}{]}
\end{sphinxVerbatim}

\end{description}

\sphinxAtStartPar
\sphinxstylestrong{Round Tag}
\begin{quote}

\sphinxAtStartPar
Examples:

\begin{sphinxVerbatim}[commandchars=\\\{\}]
\PYG{p}{[}\PYG{n}{Round} \PYG{l+s+s2}{\PYGZdq{}}\PYG{l+s+s2}{1}\PYG{l+s+s2}{\PYGZdq{}}\PYG{p}{]}
\PYG{p}{[}\PYG{n}{Round} \PYG{l+s+s2}{\PYGZdq{}}\PYG{l+s+s2}{3.1}\PYG{l+s+s2}{\PYGZdq{}}\PYG{p}{]}
\PYG{p}{[}\PYG{n}{Round} \PYG{l+s+s2}{\PYGZdq{}}\PYG{l+s+s2}{4.1.2}\PYG{l+s+s2}{\PYGZdq{}}\PYG{p}{]}
\end{sphinxVerbatim}
\end{quote}

\sphinxAtStartPar
\sphinxstylestrong{White/Black Tag}
\begin{quote}

\sphinxAtStartPar
The White and Black tag are used to specify the names of the players.
The family or last name appears first.

\sphinxAtStartPar
Examples:

\begin{sphinxVerbatim}[commandchars=\\\{\}]
\PYG{p}{[}\PYG{n}{White} \PYG{l+s+s2}{\PYGZdq{}}\PYG{l+s+s2}{Wiersma, Harm}\PYG{l+s+s2}{\PYGZdq{}}\PYG{p}{]}
\PYG{p}{[}\PYG{n}{White} \PYG{l+s+s2}{\PYGZdq{}}\PYG{l+s+s2}{van der Wal, Jannes}\PYG{l+s+s2}{\PYGZdq{}}\PYG{p}{]}
\PYG{p}{[}\PYG{n}{White} \PYG{l+s+s2}{\PYGZdq{}}\PYG{l+s+s2}{Dragon v.4.0}\PYG{l+s+s2}{\PYGZdq{}}\PYG{p}{]}
\PYG{p}{[}\PYG{n}{White} \PYG{l+s+s2}{\PYGZdq{}}\PYG{l+s+s2}{Schwarzman, A.}\PYG{l+s+s2}{\PYGZdq{}}\PYG{p}{]}
\end{sphinxVerbatim}
\end{quote}
\begin{description}
\sphinxlineitem{\sphinxstylestrong{WhiteTime/BlackTime Tag}}
\sphinxAtStartPar
The WhiteTime and BlackTime tags specify the amount of time that the players
have used during the game. Note that these tags do not exist in earlier versions
of the PDN standard. It is common practice to record these times, hence it seems
logical to define a tag for it.

\sphinxAtStartPar
Clock times are specified in \sphinxcode{\sphinxupquote{{[}H{]}H:MM{[}:SS{]}}} format. Note that in practice also
\sphinxcode{\sphinxupquote{{[}H{]}H.MM{[}.SS{]}}} is used.

\sphinxAtStartPar
\sphinxstyleemphasis{When Fischer time controls are used it makes more sense to record the remaining
time on the clock. A notation is needed to specify this.}

\sphinxAtStartPar
Examples:

\begin{sphinxVerbatim}[commandchars=\\\{\}]
\PYG{p}{[}\PYG{n}{WhiteTime} \PYG{l+s+s2}{\PYGZdq{}}\PYG{l+s+s2}{1:59:20}\PYG{l+s+s2}{\PYGZdq{}}\PYG{p}{]}
\PYG{p}{[}\PYG{n}{BlackTime} \PYG{l+s+s2}{\PYGZdq{}}\PYG{l+s+s2}{1:17:28}\PYG{l+s+s2}{\PYGZdq{}}\PYG{p}{]}
\end{sphinxVerbatim}

\end{description}

\sphinxAtStartPar
\sphinxstylestrong{TimeControl Tag}
\begin{quote}

\sphinxAtStartPar
The time controls are specified using the TimeControl tag.

\sphinxAtStartPar
Time control values should match with the following grammar:

\begin{sphinxVerbatim}[commandchars=\\\{\}]
\PYG{o}{/}\PYG{o}{/} \PYG{n}{Productions}
\PYG{n}{TimeControl}   \PYG{p}{:} \PYG{n}{UNKNOWN} \PYG{o}{|} \PYG{n}{NOTIME} \PYG{o}{|} \PYG{n}{CompositeTime}
\PYG{n}{TimeElement}   \PYG{p}{:} \PYG{n}{MOVES\PYGZus{}SECONDS} \PYG{o}{|} \PYG{n}{INCREMENTAL} \PYG{o}{|} \PYG{n}{SUDDENDEATH} \PYG{o}{|} \PYG{n}{SANDCLOCK}
\PYG{n}{CompositeTime} \PYG{p}{:} \PYG{n}{TimeElement} \PYG{p}{(}\PYG{n}{COLON} \PYG{n}{TimeElement}\PYG{p}{)}\PYG{o}{*}

\PYG{o}{/}\PYG{o}{/} \PYG{n}{Tokens}
\PYG{n}{MOVES\PYGZus{}SECONDS} \PYG{p}{:} \PYG{l+s+s2}{\PYGZdq{}}\PYG{l+s+s2}{[0\PYGZhy{}9]+}\PYG{l+s+s2}{\PYGZbs{}}\PYG{l+s+s2}{/[0\PYGZhy{}9]+}\PYG{l+s+s2}{\PYGZdq{}}
\PYG{n}{INCREMENTAL}   \PYG{p}{:} \PYG{l+s+s2}{\PYGZdq{}}\PYG{l+s+s2}{[0\PYGZhy{}9]+}\PYG{l+s+s2}{\PYGZbs{}}\PYG{l+s+s2}{+[0\PYGZhy{}9]+}\PYG{l+s+s2}{\PYGZdq{}}
\PYG{n}{SUDDENDEATH}   \PYG{p}{:} \PYG{l+s+s2}{\PYGZdq{}}\PYG{l+s+s2}{[0\PYGZhy{}9]+}\PYG{l+s+s2}{\PYGZdq{}}
\PYG{n}{SANDCLOCK}     \PYG{p}{:} \PYG{l+s+s2}{\PYGZdq{}}\PYG{l+s+s2}{\PYGZbs{}}\PYG{l+s+s2}{*[0\PYGZhy{}9]+}\PYG{l+s+s2}{\PYGZdq{}}
\PYG{n}{UNKNOWN}       \PYG{p}{:} \PYG{l+s+s2}{\PYGZdq{}}\PYG{l+s+s2}{\PYGZbs{}}\PYG{l+s+s2}{?}\PYG{l+s+s2}{\PYGZdq{}}
\PYG{n}{NOTIME}        \PYG{p}{:} \PYG{l+s+s2}{\PYGZdq{}}\PYG{l+s+s2}{\PYGZbs{}}\PYG{l+s+s2}{\PYGZhy{}}\PYG{l+s+s2}{\PYGZdq{}}
\PYG{n}{COLON}         \PYG{p}{:} \PYG{l+s+s2}{\PYGZdq{}}\PYG{l+s+s2}{\PYGZbs{}}\PYG{l+s+s2}{:}\PYG{l+s+s2}{\PYGZdq{}}
\end{sphinxVerbatim}

\sphinxAtStartPar
Examples:

\begin{sphinxVerbatim}[commandchars=\\\{\}]
\PYG{p}{[}\PYG{n}{TimeControl} \PYG{l+s+s2}{\PYGZdq{}}\PYG{l+s+s2}{40/7200:3600}\PYG{l+s+s2}{\PYGZdq{}}\PYG{p}{]}          \PYG{p}{\PYGZob{}} \PYG{l+m+mi}{40} \PYG{n}{moves} \PYG{o+ow}{in} \PYG{l+m+mi}{7200} \PYG{n}{seconds}\PYG{p}{,} \PYG{l+m+mi}{3600} \PYG{n}{seconds} \PYG{k}{for} \PYG{n}{the} \PYG{n}{rest} \PYG{n}{of} \PYG{n}{the} \PYG{n}{game} \PYG{p}{\PYGZcb{}}
\PYG{p}{[}\PYG{n}{TimeControl} \PYG{l+s+s2}{\PYGZdq{}}\PYG{l+s+s2}{4800+60}\PYG{l+s+s2}{\PYGZdq{}}\PYG{p}{]}               \PYG{p}{\PYGZob{}} \PYG{l+m+mi}{80} \PYG{n}{minutes} \PYG{k}{with} \PYG{n}{increment} \PYG{n}{of} \PYG{l+m+mi}{60} \PYG{n}{seconds}\PYG{o}{/}\PYG{n}{move} \PYG{p}{\PYGZcb{}}
\PYG{p}{[}\PYG{n}{TimeControl} \PYG{l+s+s2}{\PYGZdq{}}\PYG{l+s+s2}{40/7200:3600+60}\PYG{l+s+s2}{\PYGZdq{}}\PYG{p}{]}       \PYG{p}{\PYGZob{}} \PYG{l+m+mi}{40} \PYG{n}{moves} \PYG{o+ow}{in} \PYG{l+m+mi}{2} \PYG{n}{hours}\PYG{p}{,} \PYG{l+m+mi}{1} \PYG{n}{hour} \PYG{k}{for} \PYG{n}{the} \PYG{n}{rest} \PYG{n}{of} \PYG{n}{the} \PYG{n}{game} \PYG{k}{with} \PYG{n}{increment} \PYG{n}{of} \PYG{l+m+mi}{60} \PYG{n}{seconds}\PYG{o}{/}\PYG{n}{move} \PYG{p}{\PYGZcb{}}
\PYG{p}{[}\PYG{n}{TimeControl} \PYG{l+s+s2}{\PYGZdq{}}\PYG{l+s+s2}{40/7200:20/2400:600+5}\PYG{l+s+s2}{\PYGZdq{}}\PYG{p}{]} \PYG{p}{\PYGZob{}} \PYG{l+m+mi}{40} \PYG{n}{moves} \PYG{o+ow}{in} \PYG{l+m+mi}{2} \PYG{n}{hours}\PYG{p}{,} \PYG{l+m+mi}{20} \PYG{n}{moves} \PYG{o+ow}{in} \PYG{l+m+mi}{40} \PYG{n}{minutes}\PYG{p}{,} \PYG{l+m+mi}{10} \PYG{n}{minutes} \PYG{k}{for} \PYG{n}{the} \PYG{n}{rest} \PYG{n}{of} \PYG{n}{the} \PYG{n}{game} \PYG{k}{with} \PYG{n}{increment} \PYG{n}{of} \PYG{l+m+mi}{5} \PYG{n}{seconds}\PYG{o}{/}\PYG{n}{move} \PYG{p}{\PYGZcb{}}
\PYG{p}{[}\PYG{n}{TimeControl} \PYG{l+s+s2}{\PYGZdq{}}\PYG{l+s+s2}{*120}\PYG{l+s+s2}{\PYGZdq{}}\PYG{p}{]}                  \PYG{p}{\PYGZob{}} \PYG{l+m+mi}{2} \PYG{n}{minutes} \PYG{k}{for} \PYG{n}{a} \PYG{l+s+s2}{\PYGZdq{}}\PYG{l+s+s2}{sandclock}\PYG{l+s+s2}{\PYGZdq{}} \PYG{o+ow}{or} \PYG{l+s+s2}{\PYGZdq{}}\PYG{l+s+s2}{hourglass}\PYG{l+s+s2}{\PYGZdq{}} \PYG{n}{control} \PYG{n}{period}\PYG{p}{,} \PYG{n}{more} \PYG{n}{suitable} \PYG{n}{usage} \PYG{k}{with} \PYG{n}{physical} \PYG{n}{sandclock} \PYG{p}{\PYGZcb{}}
\end{sphinxVerbatim}
\end{quote}

\sphinxstepscope


\chapter{PDN Grammars}
\label{\detokenize{grammar:pdn-grammars}}\label{\detokenize{grammar:grammar-section}}\label{\detokenize{grammar::doc}}
\sphinxAtStartPar
This section defines some PDN grammars. First a PDN 3.0 grammar is given, with some
additional restrictions. The goal of these restrictions is to make PDN easier to parse,
and thus to make it easier for programmers to support PDN 3.0.
Also some explanations are given. Finally, a much more liberal \sphinxstyleemphasis{reading} grammar is
given that can be used when reading existing PDN files.
The grammars that are given in this section can not be parsed using a simple LL(1)
parser. See the {\hyperref[\detokenize{implementation:implementation-section}]{\sphinxcrossref{\DUrole{std}{\DUrole{std-ref}{PDN Implementation}}}}} section for some examples of LL(1)
grammars.


\section{PDN 3.0 Grammar}
\label{\detokenize{grammar:pdn-3-0-grammar}}
\begin{sphinxVerbatim}[commandchars=\\\{\}]
// Game independent productions
PdnFile          : Game (GameSeparator Game)* GameSeparator?
GameSeparator    : ASTERISK
Game             : (GameHeader GameBody?) | GameBody
GameHeader       : PdnTag+
GameBody         : (GameMove | Variation | COMMENT | SETUP | NAG)+
PdnTag           : LBRACKET IDENTIFIER STRING RBRACKET
GameMove         : MOVENUMBER? Move MOVESTRENGTH?
Variation        : LPAREN GameBody RPAREN

// Game dependent productions
Move             : NormalMove | CaptureMove
NormalMove       : Square MOVESEPARATOR Square
CaptureMove      : Square (CAPTURESEPARATOR Square)+
Square           : ALPHASQUARE | NUMSQUARE

// Tokens
MOVENUMBER       : \PYGZdq{}[0\PYGZhy{}9]+\PYGZbs{}.(\PYGZbs{}.\PYGZbs{}.)?\PYGZdq{}
MOVESEPARATOR    : \PYGZdq{}\PYGZhy{}\PYGZdq{}
CAPTURESEPARATOR : \PYGZdq{}x\PYGZdq{}
ALPHASQUARE      : \PYGZdq{}[a\PYGZhy{}h][1\PYGZhy{}8]\PYGZdq{}
NUMSQUARE        : \PYGZdq{}[1\PYGZhy{}9][0\PYGZhy{}9]?\PYGZdq{}
MOVESTRENGTH     : \PYGZdq{}([\PYGZbs{}!\PYGZbs{}?]+)|(\PYGZbs{}([\PYGZbs{}!\PYGZbs{}?]+\PYGZbs{}))\PYGZdq{}
NAG              : \PYGZdq{}\PYGZbs{}\PYGZdl{}[0\PYGZhy{}9]+\PYGZdq{}
LPAREN           : \PYGZdq{}\PYGZbs{}(\PYGZdq{}
RPAREN           : \PYGZdq{}\PYGZbs{})\PYGZdq{}
LBRACKET         : \PYGZdq{}\PYGZbs{}[\PYGZdq{}
RBRACKET         : \PYGZdq{}\PYGZbs{}]\PYGZdq{}
ASTERISK         : \PYGZdq{}\PYGZbs{}*\PYGZdq{}
SETUP            : \PYGZdq{}\PYGZbs{}/[\PYGZca{}\PYGZbs{}/]*\PYGZbs{}/\PYGZdq{}
STRING           : \PYGZdq{}\PYGZbs{}\PYGZdq{}([\PYGZca{}\PYGZbs{}\PYGZdq{}]|\PYGZbs{}\PYGZbs{}\PYGZbs{}\PYGZdq{})*\PYGZbs{}\PYGZdq{}\PYGZdq{}
COMMENT          : \PYGZdq{}\PYGZbs{}\PYGZob{}[\PYGZca{}\PYGZcb{}]*\PYGZbs{}\PYGZcb{}\PYGZdq{}
IDENTIFIER       : \PYGZdq{}[A\PYGZhy{}Z][a\PYGZhy{}zA\PYGZhy{}Z0\PYGZhy{}9\PYGZus{}]*\PYGZdq{}
\end{sphinxVerbatim}

\sphinxAtStartPar
Besides the usual white space characters (spaces, tabs and line endings),
also line comments starting with a \sphinxcode{\sphinxupquote{\%}}  are allowed. For example

\begin{sphinxVerbatim}[commandchars=\\\{\}]
\PYG{o}{\PYGZpc{}} \PYG{n}{Board} \PYG{n}{game}\PYG{p}{:} \PYG{n}{International} \PYG{n}{Draughts} \PYG{l+m+mi}{10}\PYG{n}{x10}
\PYG{p}{[}\PYG{n}{Date} \PYG{l+s+s2}{\PYGZdq{}}\PYG{l+s+s2}{2012.02.01}\PYG{l+s+s2}{\PYGZdq{}}\PYG{p}{]}
\PYG{l+m+mf}{1.} \PYG{l+m+mi}{32}\PYG{o}{\PYGZhy{}}\PYG{l+m+mi}{28} \PYG{l+m+mi}{19}\PYG{o}{\PYGZhy{}}\PYG{l+m+mi}{23}
\end{sphinxVerbatim}


\section{PDN 3.0 Restrictions}
\label{\detokenize{grammar:pdn-3-0-restrictions}}
\sphinxAtStartPar
When writing PDN, the following restrictions should be applied:
\begin{enumerate}
\sphinxsetlistlabels{\arabic}{enumi}{enumii}{}{.}%
\item {} 
\sphinxAtStartPar
Spaces are not allowed in the notation of a move. For example, \sphinxcode{\sphinxupquote{1 \sphinxhyphen{} 7}} is not allowed.

\item {} 
\sphinxAtStartPar
Spaces are not allowed between a move notation and it’s move strength indicator. For example, \sphinxcode{\sphinxupquote{32\sphinxhyphen{}28 !}} is not allowed.

\item {} 
\sphinxAtStartPar
The symbol \sphinxcode{\sphinxupquote{*}} is not allowed as a move strength indicator. Use the \sphinxcode{\sphinxupquote{\$7}} numeric annotation glyph instead.
See the {\hyperref[\detokenize{issues:issues-section}]{\sphinxcrossref{\DUrole{std}{\DUrole{std-ref}{PDN parsing issues}}}}} section for an explanation.

\item {} 
\sphinxAtStartPar
Only the symbol \sphinxcode{\sphinxupquote{*}} is allowed as a game separator.

\item {} 
\sphinxAtStartPar
Squares of moves may not have leading zeroes. For example \sphinxcode{\sphinxupquote{01\sphinxhyphen{}07}} is not allowed.

\item {} 
\sphinxAtStartPar
Moves must be written in the format (Alpha numeric, Numeric or SAN) as it is specified in the Notation attribute of the
\sphinxcode{\sphinxupquote{GameType}} tag. See also section {\hyperref[\detokenize{gametype:gametype-section}]{\sphinxcrossref{\DUrole{std}{\DUrole{std-ref}{GameType tag}}}}}.

\item {} 
\sphinxAtStartPar
Capture moves must be written using the capture separator corresponding to the \sphinxcode{\sphinxupquote{GameType}}, as it is specified in the
table in section {\hyperref[\detokenize{gametype:gametype-section}]{\sphinxcrossref{\DUrole{std}{\DUrole{std-ref}{GameType tag}}}}}.

\item {} 
\sphinxAtStartPar
Ambiguous moves must be written in long notation, i.e. they must contain the full capture sequence. All other moves
should use regular notation, i.e. only a begin and an end square.

\sphinxAtStartPar
Explanation: sometimes the regular notation of a move is ambiguous. For example in the position below the notation
\sphinxcode{\sphinxupquote{47x36}} does not specify exactly which black pieces were captured.

\noindent\sphinxincludegraphics{{diagram1}.png}

\sphinxAtStartPar
To resolve this, \sphinxcode{\sphinxupquote{47x38x24x13x36}} or \sphinxcode{\sphinxupquote{47x38x20x9x36}} must be chosen.

\item {} 
\sphinxAtStartPar
Disambiguated capture sequences have to specify a complete sequence of intermediate squares along the path of the
capture. If there is a change in direction, an intermediate square is the square where a turn in direction was made.
If there was not a change in direction, the intermediate square is the square immediately behind a captured piece.
There is no intermediate square behind the last captured piece, but otherwise leaving out an intermediate square
that is not necessary for the disambiguation is forbidden.

\sphinxAtStartPar
For example, in the above diagram \sphinxcode{\sphinxupquote{47x24x36}} is not allowed, even though it uniquely determines the move.
Also \sphinxcode{\sphinxupquote{47x33x24x13x36}} is not allowed, since 33 is not immediately behind a captured piece.

\end{enumerate}

\sphinxAtStartPar
N.B. The first five restrictions are enforced by the grammar. Other restrictions have to be checked after parsing.


\section{Explanation}
\label{\detokenize{grammar:explanation}}
\sphinxAtStartPar
The grammar is given in \sphinxhref{http://en.wikipedia.org/wiki/Extended\_Backus\%E2\%80\%93Naur\_Form}{EBNF} format.
In the productions the symbol \sphinxcode{\sphinxupquote{?}} stands for 0 or 1 repetitions, \sphinxcode{\sphinxupquote{*}} stands for 0 or more repetitions,
and \sphinxcode{\sphinxupquote{+}} stands for 1 or more repetitions. Tokens are given between double quotes and should be interpreted
as \sphinxhref{http://en.wikipedia.org/wiki/Regular\_expression}{regular expressions}.
\begin{itemize}
\item {} 
\sphinxAtStartPar
Strings can have embedded double quotes \sphinxcode{\sphinxupquote{"}}, by using the escape sequence \sphinxcode{\sphinxupquote{\textbackslash{}"}}. For example \sphinxcode{\sphinxupquote{"An embedded \textbackslash{}" quote!"}}.

\item {} 
\sphinxAtStartPar
Comments are placed between braces. For example \sphinxcode{\sphinxupquote{\{ Start of the game \} 33\sphinxhyphen{}28 18\sphinxhyphen{}22 39\sphinxhyphen{}33? \{ This is a classical mistake \}}}.

\item {} 
\sphinxAtStartPar
In existing PDN files games are usually terminated with a result. It can be one of the chess results
\sphinxcode{\sphinxupquote{1\sphinxhyphen{}0, 1/2\sphinxhyphen{}1/2, 0\sphinxhyphen{}1}}, one of the results of international draughts \sphinxcode{\sphinxupquote{2\sphinxhyphen{}0, 1\sphinxhyphen{}1, 0\sphinxhyphen{}2}}, or a
double forfeit \sphinxcode{\sphinxupquote{0\sphinxhyphen{}0}}. Finally the \sphinxcode{\sphinxupquote{*}} can be used as a terminator.

\item {} 
\sphinxAtStartPar
A game can not be empty.

\item {} 
\sphinxAtStartPar
Both numeric moves \sphinxcode{\sphinxupquote{32\sphinxhyphen{}28}} and alpha\sphinxhyphen{}numeric moves \sphinxcode{\sphinxupquote{a3\sphinxhyphen{}b4}} are allowed.

\item {} 
\sphinxAtStartPar
In alpha\sphinxhyphen{}numeric moves the separator may be omitted, so \sphinxcode{\sphinxupquote{a3b4}} is allowed.

\item {} 
\sphinxAtStartPar
A move number is a number followed by either one dot or three dots, for example \sphinxcode{\sphinxupquote{1. 32\sphinxhyphen{}28}} or
\sphinxcode{\sphinxupquote{23... 20\sphinxhyphen{}25}}. The three dots denotes that it is a black move.

\item {} 
\sphinxAtStartPar
Moves can be annotated using a move strength indicator right after their notation, for example \sphinxcode{\sphinxupquote{10\sphinxhyphen{}15!}} or \sphinxcode{\sphinxupquote{29\sphinxhyphen{}23(?)}}.

\item {} 
\sphinxAtStartPar
Moves can also be annotated using \sphinxhref{http://en.wikipedia.org/wiki/Numeric\_Annotation\_Glyphs}{numeric annotation glyphs}
(NAGs). For example \sphinxcode{\sphinxupquote{\$1}} has the same meaning as the move strength indicator \sphinxcode{\sphinxupquote{!}}.

\item {} 
\sphinxAtStartPar
Comments, variations and NAGs may appear anywhere in the game, in any order. This is less
restrictive than in \sphinxcite{introduction:nemesis},

\item {} 
\sphinxAtStartPar
Variations are placed between parentheses. They can be nested arbitrarily, for example:
\sphinxcode{\sphinxupquote{32\sphinxhyphen{}28 19\sphinxhyphen{}23 (18\sphinxhyphen{}23 38\sphinxhyphen{}32 (37\sphinxhyphen{}32? 23\sphinxhyphen{}29! \{ Black wins \}) 12\sphinxhyphen{}18) 28x19 14x23}}.

\item {} 
\sphinxAtStartPar
A setup of a new position is a FEN command surrounded by forward slashes (\sphinxcode{\sphinxupquote{/}}), and it can appear
anywhere in the game. An example is \sphinxcode{\sphinxupquote{/FEN "B:W18,24,27,28,K10,K15:B12,16,20,K22,K25,K29"/}}.
It is not backward compatible with existing PDN, see the {\hyperref[\detokenize{extensions:extensions-section}]{\sphinxcrossref{\DUrole{std}{\DUrole{std-ref}{PDN Extensions}}}}} section
for an explanation.

\end{itemize}


\section{PDN Reading Grammar}
\label{\detokenize{grammar:pdn-reading-grammar}}
\begin{sphinxVerbatim}[commandchars=\\\{\}]
// Game independent productions
PdnFile          : Game (GameSeparator Game)* GameSeparator?
GameSeparator    : ASTERISK | Result
Game             : (GameHeader GameBody?) | GameBody
GameHeader       : PdnTag+
GameBody         : (GameMove | Variation | COMMENT | SETUP | NAG)+
PdnTag           : LBRACKET IDENTIFIER STRING RBRACKET
GameMove         : MOVENUMBER? Move MOVESTRENGTH?
Variation        : LPAREN GameBody RPAREN

// Game dependent productions
Move             : NormalMove | CaptureMove | ELLIPSES
NormalMove       : Square MOVESEPARATOR Square
CaptureMove      : Square (CAPTURESEPARATOR Square)+
Square           : ALPHASQUARE | NUMSQUARE
Result           : Result1 | Result2 | DOUBLEFORFEIT
Result1          : WIN1 | DRAW1 | LOSS1
Result2          : WIN2 | DRAW2 | LOSS2

// Tokens
WIN1             : \PYGZdq{}1\PYGZhy{}0\PYGZdq{}
DRAW1            : \PYGZdq{}1\PYGZbs{}/2\PYGZhy{}1\PYGZbs{}/2\PYGZdq{}
LOSS1            : \PYGZdq{}0\PYGZhy{}1\PYGZdq{}
WIN2             : \PYGZdq{}2\PYGZhy{}0\PYGZdq{}
DRAW2            : \PYGZdq{}1\PYGZhy{}1\PYGZdq{}
LOSS2            : \PYGZdq{}0\PYGZhy{}2\PYGZdq{}
DOUBLEFORFEIT    : \PYGZdq{}0\PYGZhy{}0\PYGZdq{}
ELLIPSES         : \PYGZdq{}\PYGZbs{}.\PYGZbs{}.\PYGZbs{}.\PYGZdq{}
MOVENUMBER       : \PYGZdq{}[0\PYGZhy{}9]+\PYGZbs{}.(\PYGZbs{}.\PYGZbs{}.)?\PYGZdq{}
MOVESEPARATOR    : \PYGZdq{}\PYGZhy{}\PYGZdq{}
CAPTURESEPARATOR : \PYGZdq{}[x:]\PYGZdq{}
ALPHASQUARE      : \PYGZdq{}[a\PYGZhy{}h][1\PYGZhy{}8]\PYGZdq{}
NUMSQUARE        : \PYGZdq{}([1\PYGZhy{}9][0\PYGZhy{}9]?)|(0[1\PYGZhy{}9])\PYGZdq{}
MOVESTRENGTH     : \PYGZdq{}([\PYGZbs{}!\PYGZbs{}?]+)|(\PYGZbs{}([\PYGZbs{}!\PYGZbs{}?]+\PYGZbs{}))\PYGZdq{}
NAG              : \PYGZdq{}\PYGZbs{}\PYGZdl{}[0\PYGZhy{}9]+\PYGZdq{}
LPAREN           : \PYGZdq{}\PYGZbs{}(\PYGZdq{}
RPAREN           : \PYGZdq{}\PYGZbs{})\PYGZdq{}
LBRACKET         : \PYGZdq{}\PYGZbs{}[\PYGZdq{}
RBRACKET         : \PYGZdq{}\PYGZbs{}]\PYGZdq{}
ASTERISK         : \PYGZdq{}\PYGZbs{}*\PYGZdq{}
SETUP            : \PYGZdq{}\PYGZbs{}/[\PYGZca{}\PYGZbs{}/]*\PYGZbs{}/\PYGZdq{}
STRING           : \PYGZdq{}\PYGZbs{}\PYGZdq{}([\PYGZca{}\PYGZbs{}\PYGZdq{}]|\PYGZbs{}\PYGZbs{}\PYGZbs{}\PYGZdq{})*\PYGZbs{}\PYGZdq{}\PYGZdq{}
COMMENT          : \PYGZdq{}\PYGZbs{}\PYGZob{}[\PYGZca{}\PYGZcb{}]*\PYGZbs{}\PYGZcb{}\PYGZdq{}
IDENTIFIER       : \PYGZdq{}[A\PYGZhy{}Z][a\PYGZhy{}zA\PYGZhy{}Z0\PYGZhy{}9\PYGZus{}]*\PYGZdq{}
\end{sphinxVerbatim}

\sphinxAtStartPar
When reading PDN, one must take into account that captures can can contain the full capture sequence,
also in the case of non\sphinxhyphen{}ambiguous moves. This is for backward compatibility.

\sphinxAtStartPar
Note that the above \sphinxstyleemphasis{reading} grammar does not take everything into account. Some additional
constructs that are encountered in practice are:
\begin{itemize}
\item {} 
\sphinxAtStartPar
Specify unknown moves using a minus sign: \sphinxcode{\sphinxupquote{41. \sphinxhyphen{} \sphinxhyphen{}}}.

\item {} 
\sphinxAtStartPar
Write a move number without a move to specify an empty game: \sphinxcode{\sphinxupquote{1. 2\sphinxhyphen{}0}}.

\item {} 
\sphinxAtStartPar
Omit move separators in numeric moves: \sphinxcode{\sphinxupquote{3228 1823}}.

\end{itemize}

\sphinxAtStartPar
The reading grammar can be extended to handle such cases, but this goes beyond the
scope of this document.

\sphinxAtStartPar
In several PDN examples three dots are being used for an unspecified or unknown move.
For example \sphinxcode{\sphinxupquote{1. ... 7\sphinxhyphen{}12}} is sometimes used instead of \sphinxcode{\sphinxupquote{1... 7\sphinxhyphen{}12}}. This case has
been added to the reading grammar by means of the \sphinxcode{\sphinxupquote{ELLIPSES}} token.

\sphinxstepscope


\chapter{PDN Extensions}
\label{\detokenize{extensions:pdn-extensions}}\label{\detokenize{extensions:extensions-section}}\label{\detokenize{extensions::doc}}

\section{Embedded commands}
\label{\detokenize{extensions:embedded-commands}}
\sphinxAtStartPar
Comments in PDN files may have embedded commands. Embedded
commands may appear anywhere inside comments, and they have the
following syntax: \sphinxcode{\sphinxupquote{{[}\%COMMAND VALUE{]}}}, where \sphinxcode{\sphinxupquote{COMMAND}} identifies
the command and \sphinxcode{\sphinxupquote{VALUE}} is command\sphinxhyphen{}specific syntax.


\subsection{clk, mct, egt and emt command}
\label{\detokenize{extensions:clk-mct-egt-and-emt-command}}
\sphinxAtStartPar
The following commands are taken from \sphinxcite{introduction:dgt}:


\begin{savenotes}\sphinxattablestart
\sphinxthistablewithglobalstyle
\centering
\begin{tabulary}{\linewidth}[t]{TT}
\sphinxtoprule
\sphinxstyletheadfamily 
\sphinxAtStartPar
Command
&\sphinxstyletheadfamily 
\sphinxAtStartPar
Description
\\
\sphinxmidrule
\sphinxtableatstartofbodyhook
\sphinxAtStartPar
clk
&
\sphinxAtStartPar
Time displayed on a clock (remaining time)
\\
\sphinxhline
\sphinxAtStartPar
mct
&
\sphinxAtStartPar
Time displayed on a clock (elapsed time)
\\
\sphinxhline
\sphinxAtStartPar
egt
&
\sphinxAtStartPar
Elapsed game time
\\
\sphinxhline
\sphinxAtStartPar
emt
&
\sphinxAtStartPar
Elapsed move time
\\
\sphinxbottomrule
\end{tabulary}
\sphinxtableafterendhook\par
\sphinxattableend\end{savenotes}

\sphinxAtStartPar
The values of the \sphinxcode{\sphinxupquote{clk}}, \sphinxcode{\sphinxupquote{egt}}, \sphinxcode{\sphinxupquote{emt}} and \sphinxcode{\sphinxupquote{mct}} commands are in \sphinxcode{\sphinxupquote{h:mm:ss}} format.
Usually \sphinxcode{\sphinxupquote{clk}} values come from a digital clock, while \sphinxcode{\sphinxupquote{mct}} values come from
a mechanical clock.

\sphinxAtStartPar
\sphinxstylestrong{Examples}

\begin{sphinxVerbatim}[commandchars=\\\{\}]
\PYG{l+m+mf}{1.} \PYG{l+m+mi}{31}\PYG{o}{\PYGZhy{}}\PYG{l+m+mi}{26} \PYG{p}{\PYGZob{}}\PYG{p}{[}\PYG{o}{\PYGZpc{}}\PYG{n}{clk} \PYG{l+m+mi}{1}\PYG{p}{:}\PYG{l+m+mi}{55}\PYG{p}{:}\PYG{l+m+mi}{21}\PYG{p}{]}\PYG{p}{\PYGZcb{}}
\end{sphinxVerbatim}


\subsection{clock command}
\label{\detokenize{extensions:clock-command}}
\sphinxAtStartPar
The \sphinxcode{\sphinxupquote{clock}} command is an extension of the \sphinxcode{\sphinxupquote{clk}} command.
An embedded \sphinxcode{\sphinxupquote{clk}} command should match with the following regular expression:

\begin{sphinxVerbatim}[commandchars=\\\{\}]
\PYGZbs{}[\PYGZpc{}clock\PYGZbs{}s*(([wWbB])(\PYGZbs{}d\PYGZob{}1,2\PYGZcb{}:\PYGZbs{}d\PYGZbs{}d:\PYGZbs{}d\PYGZbs{}d)\PYGZbs{}s*)(([wWbB])(\PYGZbs{}d\PYGZob{}1,2\PYGZcb{}:\PYGZbs{}d\PYGZbs{}d:\PYGZbs{}d\PYGZbs{}d)\PYGZbs{}s*)?\PYGZbs{}]
\end{sphinxVerbatim}

\sphinxAtStartPar
\sphinxstylestrong{Examples}

\begin{sphinxVerbatim}[commandchars=\\\{\}]
\PYG{l+m+mf}{1.} \PYG{l+m+mi}{31}\PYG{o}{\PYGZhy{}}\PYG{l+m+mi}{26} \PYG{p}{\PYGZob{}}\PYG{p}{[}\PYG{o}{\PYGZpc{}}\PYG{n}{clk} \PYG{n}{w0}\PYG{p}{:}\PYG{l+m+mi}{00}\PYG{p}{:}\PYG{l+m+mi}{10} \PYG{n}{B0}\PYG{p}{:}\PYG{l+m+mi}{00}\PYG{p}{:}\PYG{l+m+mi}{03}\PYG{p}{]}\PYG{p}{\PYGZcb{}}

\PYG{l+m+mf}{23.44}\PYG{o}{\PYGZhy{}}\PYG{l+m+mi}{39} \PYG{p}{\PYGZob{}}\PYG{p}{[}\PYG{o}{\PYGZpc{}}\PYG{n}{clk} \PYG{l+m+mi}{1}\PYG{p}{:}\PYG{l+m+mi}{05}\PYG{p}{:}\PYG{l+m+mi}{23}\PYG{p}{]}\PYG{p}{\PYGZcb{}} \PYG{l+m+mi}{18}\PYG{o}{\PYGZhy{}}\PYG{l+m+mi}{23} \PYG{p}{\PYGZob{}}\PYG{n}{Optional} \PYG{n}{leading} \PYG{n}{comments}
\PYG{p}{[}\PYG{o}{\PYGZpc{}}\PYG{n}{clock} \PYG{l+m+mi}{0}\PYG{p}{:}\PYG{l+m+mi}{49}\PYG{p}{:}\PYG{l+m+mi}{11}\PYG{p}{]} \PYG{n}{optional} \PYG{n}{trailing} \PYG{n}{comments}\PYG{p}{\PYGZcb{}}
\end{sphinxVerbatim}

\sphinxAtStartPar
In cases like this the clock time is connected to the preceding move. It is
the preferred way of specifying clock times during live recordings with
electronic boards. A clock command may contain one or two clock times. Each
of them may be preceded by a \sphinxcode{\sphinxupquote{w}} or a \sphinxcode{\sphinxupquote{b}} to denote the clock time for
white or black. If uppercase is used, it means that the clock for this player
is running.


\section{Setup commands}
\label{\detokenize{extensions:setup-commands}}
\sphinxAtStartPar
The PDN grammars presented in this document contain an extension for doing
setups of a position anywhere in the game. The motivation for having this
command is threefold:
\begin{itemize}
\item {} 
\sphinxAtStartPar
It gives a well\sphinxhyphen{}defined way to handle illegal moves, that happen occasionally
in tournament practice.

\item {} 
\sphinxAtStartPar
They can be used to handle move recognition failures of electronic board
software.

\item {} 
\sphinxAtStartPar
In game analysis it is common to make side steps to different positions,
for example to a similar position that has occurred before. The setup
command allows to incorporate these side steps as normal variations
starting with a setup.

\end{itemize}

\sphinxAtStartPar
A setup command is a FEN setup command surrounded by forward slashes.
Note that this extension is not backward compatible with older versions
of the PDN standard. This is an intentional choice, since the moves which
appear after a setup are ill\sphinxhyphen{}defined if the setup is ignored. Setups are changes
on the board, and so they should be on the same level as moves. It is therefore
not a good idea to model them as embedded commands inside comments.

\sphinxAtStartPar
\sphinxstylestrong{Null moves}

\sphinxAtStartPar
For programmers, null moves are sometimes useful to denote an empty move.
A possible notation for a null move, proposed by Gérard Taille, is

\begin{sphinxVerbatim}[commandchars=\\\{\}]
\PYG{o}{/}\PYG{n}{FEN} \PYG{l+s+s2}{\PYGZdq{}}\PYG{l+s+s2}{B::}\PYG{l+s+s2}{\PYGZdq{}}\PYG{o}{/}
\end{sphinxVerbatim}

\sphinxAtStartPar
\sphinxstylestrong{Examples}

\begin{sphinxVerbatim}[commandchars=\\\{\}]
\PYG{l+m+mf}{1.31}\PYG{o}{\PYGZhy{}}\PYG{l+m+mi}{26} \PYG{l+m+mi}{17}\PYG{o}{\PYGZhy{}}\PYG{l+m+mi}{21} \PYG{o}{/}\PYG{n}{FEN} \PYG{l+s+s2}{\PYGZdq{}}\PYG{l+s+s2}{B:B1\PYGZhy{}16,18\PYGZhy{}21:W26,28,33\PYGZhy{}50}\PYG{l+s+s2}{\PYGZdq{}}\PYG{o}{/}
\PYG{p}{\PYGZob{}} \PYG{n}{White} \PYG{n}{forgot} \PYG{n}{to} \PYG{n}{make} \PYG{n}{a} \PYG{n}{capture} \PYG{o+ow}{and} \PYG{n}{played} \PYG{l+m+mi}{32}\PYG{o}{\PYGZhy{}}\PYG{l+m+mi}{28} \PYG{n}{instead} \PYG{p}{\PYGZcb{}}
\end{sphinxVerbatim}

\sphinxAtStartPar
Suppose a game starts with an illegal move, resulting in the following sequence
of positions:
\begin{quote}

\noindent\sphinxincludegraphics{{setup1}.png}

\noindent\sphinxincludegraphics{{setup2}.png}

\noindent\sphinxincludegraphics{{setup3}.png}
\end{quote}

\sphinxAtStartPar
This can be encoded using

\begin{sphinxVerbatim}[commandchars=\\\{\}]
\PYG{o}{/}\PYG{n}{FEN} \PYG{l+s+s2}{\PYGZdq{}}\PYG{l+s+s2}{W:W31\PYGZhy{}50:B1\PYGZhy{}20}\PYG{l+s+s2}{\PYGZdq{}}\PYG{o}{/}
\PYG{o}{/}\PYG{n}{FEN} \PYG{l+s+s2}{\PYGZdq{}}\PYG{l+s+s2}{B:W28,31,33\PYGZhy{}50:B1\PYGZhy{}18,20}\PYG{l+s+s2}{\PYGZdq{}}\PYG{o}{/}
\PYG{l+m+mf}{01.}\PYG{o}{.}\PYG{o}{.} \PYG{l+m+mi}{14}\PYG{o}{\PYGZhy{}}\PYG{l+m+mi}{19}
\end{sphinxVerbatim}

\sphinxAtStartPar
The setup of the initial position is required here. If it was
omitted, the second setup would be taken as initial position of the game.

\sphinxstepscope


\chapter{PDN 3.1 Proposals}
\label{\detokenize{pdnnext:pdn-3-1-proposals}}\label{\detokenize{pdnnext::doc}}
\sphinxAtStartPar
This section describes proposals for the next version of the PDN standard.


\section{Alternative move disambiguation}
\label{\detokenize{pdnnext:alternative-move-disambiguation}}
\sphinxAtStartPar
The current way of move disambiguation (see section {\hyperref[\detokenize{grammar:grammar-section}]{\sphinxcrossref{\DUrole{std}{\DUrole{std-ref}{PDN Grammars}}}}}) was chosen for backward compatibility.
The following alternative is proposed. An ambiguous move is written as usual, followed by the sequence of
captured squares between angular brackets \sphinxcode{\sphinxupquote{\textless{}}} and \sphinxcode{\sphinxupquote{\textgreater{}}}. For example, in the following position
\begin{quote}

\noindent\sphinxincludegraphics{{diagram1}.png}
\end{quote}

\sphinxAtStartPar
the two possible captures are written as \sphinxcode{\sphinxupquote{47x36 \textless{}42, 29, 14, 31\textgreater{}}} and \sphinxcode{\sphinxupquote{47x36 \textless{}42, 29, 19, 31\textgreater{}}}. The
motivation for this is that it is less complicated to implement, and easy to understand for humans.

\sphinxstepscope


\chapter{PDN Implementation}
\label{\detokenize{implementation:pdn-implementation}}\label{\detokenize{implementation:implementation-section}}\label{\detokenize{implementation::doc}}
\sphinxAtStartPar
This section contains concrete examples of PDN grammars.


\section{DParser}
\label{\detokenize{implementation:dparser}}
\sphinxAtStartPar
\sphinxcite{implementation:id4} is a C parser generator.
\begin{itemize}
\item {} 
\sphinxAtStartPar
\sphinxcode{\sphinxupquote{pdn\_reading.g}} A fairly liberal reading grammar.

\item {} 
\sphinxAtStartPar
\sphinxcode{\sphinxupquote{pdn\_writing.g}} A PDN 3.0 writing grammar.

\item {} 
\sphinxAtStartPar
\sphinxcode{\sphinxupquote{fen.g}} A grammar for FEN strings.

\item {} 
\sphinxAtStartPar
\sphinxcode{\sphinxupquote{timecontrol.g}} A grammar for time controls.

\end{itemize}


\section{Grammatica}
\label{\detokenize{implementation:grammatica}}
\sphinxAtStartPar
\sphinxcite{implementation:id5} is a java parser generator.
\begin{itemize}
\item {} 
\sphinxAtStartPar
\sphinxcode{\sphinxupquote{pdn\_reading.grammar}} A fairly liberal reading grammar.

\item {} 
\sphinxAtStartPar
\sphinxcode{\sphinxupquote{pdn\_writing.grammar}} A PDN 3.0 writing grammar.

\item {} 
\sphinxAtStartPar
\sphinxcode{\sphinxupquote{fen.grammar}} A grammar for FEN strings.

\item {} 
\sphinxAtStartPar
\sphinxcode{\sphinxupquote{timeControl.grammar}} A grammar for time controls.

\end{itemize}
\begin{itemize}
\item {} 
\sphinxAtStartPar
The Grammatica grammars are LL(1) grammars. They define a move as a token to make this possible.

\item {} 
\sphinxAtStartPar
The Grammatica grammars contain a workaround for move strengths, since the regular expressions in Grammatica do not behave correctly.

\end{itemize}


\section{Toy Parser Generator}
\label{\detokenize{implementation:toy-parser-generator}}
\sphinxAtStartPar
\sphinxcite{implementation:tpg} is a python parser generator.
\begin{itemize}
\item {} 
\sphinxAtStartPar
\sphinxcode{\sphinxupquote{pdn\_reading\_tpg.py}} A fairly liberal reading grammar.

\item {} 
\sphinxAtStartPar
\sphinxcode{\sphinxupquote{pdn\_writing\_tpg.py}} A PDN 3.0 writing grammar.

\end{itemize}


\section{Test files}
\label{\detokenize{implementation:test-files}}\begin{itemize}
\item {} 
\sphinxAtStartPar
\sphinxcode{\sphinxupquote{games.zip}} A collection of PDN games used for testing the grammars.

\end{itemize}

\sphinxstepscope


\chapter{PDN Examples}
\label{\detokenize{examples:pdn-examples}}\label{\detokenize{examples::doc}}
\sphinxAtStartPar
All of the examples below have been checked with the \sphinxhref{http://10x10.org/pdn/test/index.html}{PDN 3.0 Checker}.

\sphinxAtStartPar
An example of an international draughts game in PDN 3.0 format:

\begin{sphinxVerbatim}[commandchars=\\\{\}]
\PYG{p}{[}\PYG{n}{Event} \PYG{l+s+s2}{\PYGZdq{}}\PYG{l+s+s2}{FMJD World Championship}\PYG{l+s+s2}{\PYGZdq{}}\PYG{p}{]}
\PYG{p}{[}\PYG{n}{Site} \PYG{l+s+s2}{\PYGZdq{}}\PYG{l+s+s2}{Hardenberg, NED}\PYG{l+s+s2}{\PYGZdq{}}\PYG{p}{]}
\PYG{p}{[}\PYG{n}{Date} \PYG{l+s+s2}{\PYGZdq{}}\PYG{l+s+s2}{2007.05.19}\PYG{l+s+s2}{\PYGZdq{}}\PYG{p}{]}
\PYG{p}{[}\PYG{n}{Round} \PYG{l+s+s2}{\PYGZdq{}}\PYG{l+s+s2}{7}\PYG{l+s+s2}{\PYGZdq{}}\PYG{p}{]}
\PYG{p}{[}\PYG{n}{White} \PYG{l+s+s2}{\PYGZdq{}}\PYG{l+s+s2}{Mikhalchenko,I.}\PYG{l+s+s2}{\PYGZdq{}}\PYG{p}{]}
\PYG{p}{[}\PYG{n}{Black} \PYG{l+s+s2}{\PYGZdq{}}\PYG{l+s+s2}{Ndjofang,J.}\PYG{l+s+s2}{\PYGZdq{}}\PYG{p}{]}
\PYG{p}{[}\PYG{n}{Result} \PYG{l+s+s2}{\PYGZdq{}}\PYG{l+s+s2}{0\PYGZhy{}2}\PYG{l+s+s2}{\PYGZdq{}}\PYG{p}{]}
\PYG{p}{[}\PYG{n}{GameType} \PYG{l+s+s2}{\PYGZdq{}}\PYG{l+s+s2}{20}\PYG{l+s+s2}{\PYGZdq{}}\PYG{p}{]}
\PYG{p}{[}\PYG{n}{WhiteTime} \PYG{l+s+s2}{\PYGZdq{}}\PYG{l+s+s2}{1:36}\PYG{l+s+s2}{\PYGZdq{}}\PYG{p}{]}
\PYG{p}{[}\PYG{n}{BlackTime} \PYG{l+s+s2}{\PYGZdq{}}\PYG{l+s+s2}{1:17}\PYG{l+s+s2}{\PYGZdq{}}\PYG{p}{]}

 \PYG{l+m+mf}{1.32}\PYG{o}{\PYGZhy{}}\PYG{l+m+mi}{28} \PYG{l+m+mi}{17}\PYG{o}{\PYGZhy{}}\PYG{l+m+mi}{22}  \PYG{l+m+mf}{2.28}\PYG{n}{x17} \PYG{l+m+mi}{11}\PYG{n}{x22}  \PYG{l+m+mf}{3.37}\PYG{o}{\PYGZhy{}}\PYG{l+m+mi}{32}  \PYG{l+m+mi}{6}\PYG{o}{\PYGZhy{}}\PYG{l+m+mi}{11}  \PYG{l+m+mf}{4.41}\PYG{o}{\PYGZhy{}}\PYG{l+m+mi}{37} \PYG{l+m+mi}{12}\PYG{o}{\PYGZhy{}}\PYG{l+m+mi}{17}  \PYG{l+m+mf}{5.46}\PYG{o}{\PYGZhy{}}\PYG{l+m+mi}{41}  \PYG{l+m+mi}{8}\PYG{o}{\PYGZhy{}}\PYG{l+m+mi}{12}
 \PYG{l+m+mf}{6.34}\PYG{o}{\PYGZhy{}}\PYG{l+m+mi}{30}  \PYG{l+m+mi}{2}\PYG{o}{\PYGZhy{}}\PYG{l+m+mi}{8}   \PYG{l+m+mf}{7.30}\PYG{o}{\PYGZhy{}}\PYG{l+m+mi}{25} \PYG{l+m+mi}{19}\PYG{o}{\PYGZhy{}}\PYG{l+m+mi}{23}  \PYG{l+m+mf}{8.35}\PYG{o}{\PYGZhy{}}\PYG{l+m+mi}{30}  \PYG{l+m+mi}{1}\PYG{o}{\PYGZhy{}}\PYG{l+m+mi}{6}   \PYG{l+m+mf}{9.40}\PYG{o}{\PYGZhy{}}\PYG{l+m+mi}{35} \PYG{l+m+mi}{13}\PYG{o}{\PYGZhy{}}\PYG{l+m+mi}{19} \PYG{l+m+mf}{10.31}\PYG{o}{\PYGZhy{}}\PYG{l+m+mi}{27} \PYG{l+m+mi}{22}\PYG{n}{x31}
\PYG{l+m+mf}{11.36}\PYG{n}{x27}  \PYG{l+m+mi}{9}\PYG{o}{\PYGZhy{}}\PYG{l+m+mi}{13} \PYG{l+m+mf}{12.33}\PYG{o}{\PYGZhy{}}\PYG{l+m+mi}{28}  \PYG{l+m+mi}{4}\PYG{o}{\PYGZhy{}}\PYG{l+m+mi}{9}  \PYG{l+m+mf}{13.41}\PYG{o}{\PYGZhy{}}\PYG{l+m+mi}{36} \PYG{l+m+mi}{17}\PYG{o}{\PYGZhy{}}\PYG{l+m+mi}{22} \PYG{l+m+mf}{14.28}\PYG{n}{x17} \PYG{l+m+mi}{11}\PYG{n}{x31} \PYG{l+m+mf}{15.37}\PYG{n}{x26} \PYG{l+m+mi}{23}\PYG{o}{\PYGZhy{}}\PYG{l+m+mi}{28}
\PYG{l+m+mf}{16.32}\PYG{n}{x23} \PYG{l+m+mi}{19}\PYG{n}{x28} \PYG{l+m+mf}{17.42}\PYG{o}{\PYGZhy{}}\PYG{l+m+mi}{37} \PYG{l+m+mi}{20}\PYG{o}{\PYGZhy{}}\PYG{l+m+mi}{24} \PYG{l+m+mf}{18.30}\PYG{n}{x19} \PYG{l+m+mi}{14}\PYG{n}{x23} \PYG{l+m+mf}{19.37}\PYG{o}{\PYGZhy{}}\PYG{l+m+mi}{31} \PYG{l+m+mi}{16}\PYG{o}{\PYGZhy{}}\PYG{l+m+mi}{21} \PYG{l+m+mf}{20.26}\PYG{n}{x17} \PYG{l+m+mi}{12}\PYG{n}{x21}
\PYG{l+m+mf}{21.31}\PYG{o}{\PYGZhy{}}\PYG{l+m+mi}{27} \PYG{l+m+mi}{21}\PYG{n}{x32} \PYG{l+m+mf}{22.38}\PYG{n}{x27}  \PYG{l+m+mi}{6}\PYG{o}{\PYGZhy{}}\PYG{l+m+mi}{11} \PYG{l+m+mf}{23.47}\PYG{o}{\PYGZhy{}}\PYG{l+m+mi}{42} \PYG{l+m+mi}{15}\PYG{o}{\PYGZhy{}}\PYG{l+m+mi}{20} \PYG{l+m+mf}{24.25}\PYG{n}{x14} \PYG{l+m+mi}{10}\PYG{n}{x19} \PYG{l+m+mf}{25.39}\PYG{o}{\PYGZhy{}}\PYG{l+m+mi}{33} \PYG{l+m+mi}{28}\PYG{n}{x39}
\PYG{l+m+mf}{26.44}\PYG{n}{x33}  \PYG{l+m+mi}{8}\PYG{o}{\PYGZhy{}}\PYG{l+m+mi}{12} \PYG{l+m+mf}{27.42}\PYG{o}{\PYGZhy{}}\PYG{l+m+mi}{38} \PYG{l+m+mi}{23}\PYG{o}{\PYGZhy{}}\PYG{l+m+mi}{28} \PYG{l+m+mf}{28.33}\PYG{n}{x22} \PYG{l+m+mi}{12}\PYG{o}{\PYGZhy{}}\PYG{l+m+mi}{17} \PYG{l+m+mf}{29.49}\PYG{o}{\PYGZhy{}}\PYG{l+m+mi}{44} \PYG{l+m+mi}{17}\PYG{n}{x28} \PYG{l+m+mf}{30.38}\PYG{o}{\PYGZhy{}}\PYG{l+m+mi}{33} \PYG{l+m+mi}{28}\PYG{n}{x39}
\PYG{l+m+mf}{31.44}\PYG{n}{x33} \PYG{l+m+mi}{18}\PYG{o}{\PYGZhy{}}\PYG{l+m+mi}{22} \PYG{l+m+mf}{32.27}\PYG{n}{x18} \PYG{l+m+mi}{13}\PYG{n}{x22} \PYG{l+m+mf}{33.43}\PYG{o}{\PYGZhy{}}\PYG{l+m+mi}{38} \PYG{l+m+mi}{19}\PYG{o}{\PYGZhy{}}\PYG{l+m+mi}{23} \PYG{l+m+mf}{34.38}\PYG{o}{\PYGZhy{}}\PYG{l+m+mi}{32} \PYG{l+m+mi}{11}\PYG{o}{\PYGZhy{}}\PYG{l+m+mi}{17} \PYG{l+m+mf}{35.32}\PYG{o}{\PYGZhy{}}\PYG{l+m+mi}{27} \PYG{l+m+mi}{22}\PYG{n}{x31}
\PYG{l+m+mf}{36.36}\PYG{n}{x27}  \PYG{l+m+mi}{9}\PYG{o}{\PYGZhy{}}\PYG{l+m+mi}{13} \PYG{l+m+mf}{37.45}\PYG{o}{\PYGZhy{}}\PYG{l+m+mi}{40} \PYG{l+m+mi}{13}\PYG{o}{\PYGZhy{}}\PYG{l+m+mi}{18} \PYG{o}{*}
\end{sphinxVerbatim}

\sphinxAtStartPar
An example of an analysis in PDN 3.0 format:

\begin{sphinxVerbatim}[commandchars=\\\{\}]
[Event \PYGZdq{}nk\PYGZdq{}]
[Site \PYGZdq{}?\PYGZdq{}]
[Date \PYGZdq{}2009.04.08\PYGZdq{}]
[Round \PYGZdq{}?\PYGZdq{}]
[White \PYGZdq{}Derkx,B.\PYGZdq{}]
[Black \PYGZdq{}Meijer,Hein\PYGZdq{}]
[Result \PYGZdq{}2\PYGZhy{}0\PYGZdq{}]
[GameType \PYGZdq{}20\PYGZdq{}]
[WhiteTime \PYGZdq{}2.23\PYGZdq{}]
[BlackTime \PYGZdq{}2.40\PYGZdq{}]
[PlyCount \PYGZdq{}117\PYGZdq{}]

\PYGZob{} Beide spelers hebben volgens Turbo Dambase 3x eerder tegen elkaar
gespeeld. In het Nederlands kampioenschap van 2007 en 2008, en in de
halve finale van 2007, troffen beiden elkaar. Alle eerdere duels
eindigde in remise. \PYGZcb{} 1. 34\PYGZhy{}29 17\PYGZhy{}22 2. 32\PYGZhy{}28 \PYGZob{} Het populairste
antwoord in deze opening is 2.39\PYGZhy{}34 op ruime afstand gevolgd door
2.40\PYGZhy{}34 \PYGZcb{} 11\PYGZhy{}17 3. 37\PYGZhy{}32 6\PYGZhy{}11 4. 41\PYGZhy{}37 19\PYGZhy{}23 ( \PYGZob{} Meestal wordt eerst
\PYGZcb{} 4... 1\PYGZhy{}6 \PYGZob{} gespeeld om na \PYGZcb{} 5. 46\PYGZhy{}41 \PYGZob{} alsnog \PYGZcb{} 19\PYGZhy{}23 6. 28x19
14x34 \PYGZob{} te spelen. \PYGZcb{} ) 5. 28x19 14x34 6. 39x30 ( \PYGZob{} Over het algemeen
is slaan met \PYGZcb{} 6. 40x29 \PYGZob{} populairder. \PYGZcb{} ) 6... 13\PYGZhy{}19 7. 44\PYGZhy{}39 8\PYGZhy{}13
8. 50\PYGZhy{}44 20\PYGZhy{}25 9. 32\PYGZhy{}28 25x34 10. 40x29 16\PYGZhy{}21 11. 31\PYGZhy{}26 21\PYGZhy{}27
12. 37\PYGZhy{}32 11\PYGZhy{}16 13. 32x21 16x27 14. 38\PYGZhy{}32 27x38 15. 43x32 10\PYGZhy{}14
16. 49\PYGZhy{}43 5\PYGZhy{}10 17. 42\PYGZhy{}38 3\PYGZhy{}8 18. 47\PYGZhy{}42 19\PYGZhy{}23 19. 28x19 14x34
20. 39x30 13\PYGZhy{}19 21. 43\PYGZhy{}39 10\PYGZhy{}14 22. 46\PYGZhy{}41 18\PYGZhy{}23 23. 41\PYGZhy{}37 12\PYGZhy{}18
24. 37\PYGZhy{}31 7\PYGZhy{}12 25. 33\PYGZhy{}28 22x33 26. 39x28 15\PYGZhy{}20 27. 31\PYGZhy{}27 2\PYGZhy{}7
28. 44\PYGZhy{}39 7\PYGZhy{}11 29. 39\PYGZhy{}33 20\PYGZhy{}24 30. 42\PYGZhy{}37 1\PYGZhy{}6 31. 48\PYGZhy{}42 9\PYGZhy{}13 32. 30\PYGZhy{}25
23\PYGZhy{}29 33. 27\PYGZhy{}21 11\PYGZhy{}16 34. 37\PYGZhy{}31 16x27 35. 31x11 6x17 36. 45\PYGZhy{}40 17\PYGZhy{}21
37. 26x17 12x21 38. 42\PYGZhy{}37 18\PYGZhy{}23 39. 36\PYGZhy{}31 21\PYGZhy{}26 40. 31\PYGZhy{}27 8\PYGZhy{}12
41. 40\PYGZhy{}34 29x40 42. 35x44 4\PYGZhy{}10 43. 44\PYGZhy{}39 23\PYGZhy{}29 44. 28\PYGZhy{}22 13\PYGZhy{}18 ( \PYGZob{}
Zwart had hier \PYGZcb{} 44... 12\PYGZhy{}18 \PYGZob{} moeten spelen \PYGZcb{} 45. 22\PYGZhy{}17 29\PYGZhy{}34
46. 39x30 24x35 \PYGZob{} enz. \PYGZcb{} ) 45. 22x13 19x8 46. 33\PYGZhy{}28 \PYGZdl{}1 8\PYGZhy{}13 47. 28\PYGZhy{}22
13\PYGZhy{}19 \PYGZdl{}2 ( \PYGZob{} In tijdnood gaat het nodige mis, aangewezen is hier \PYGZcb{}
47... 10\PYGZhy{}15 48. 22\PYGZhy{}17 12x21 49. 27x16 13\PYGZhy{}19 \PYGZob{} en ook zwart werkt aan
zijn doorbraak. \PYGZcb{} ) 48. 25\PYGZhy{}20 \PYGZdl{}3 ( 48. 22\PYGZhy{}17 12x21 49. 27x16 26\PYGZhy{}31
50. 37x26 24\PYGZhy{}30 51. 25x23 19x37 ) 48... 24x15 49. 22\PYGZhy{}17 12x21
50. 27x16 15\PYGZhy{}20 51. 16\PYGZhy{}11 20\PYGZhy{}24 52. 11\PYGZhy{}7 29\PYGZhy{}34 53. 39x30 24x35
54. 7\PYGZhy{}1 14\PYGZhy{}20 55. 32\PYGZhy{}27 20\PYGZhy{}24 56. 27\PYGZhy{}22 10\PYGZhy{}15 57. 1\PYGZhy{}45 24\PYGZhy{}30
58. 22\PYGZhy{}17 19\PYGZhy{}24 59. 45\PYGZhy{}18 *
\end{sphinxVerbatim}

\sphinxAtStartPar
An example of a checkers game in PDN 3.0 format:

\begin{sphinxVerbatim}[commandchars=\\\{\}]
[Event \PYGZdq{}Double Corner Dyke\PYGZdq{}]
[Black \PYGZdq{}Jordan,A\PYGZdq{}]
[White \PYGZdq{}Tesheliet,F\PYGZdq{}]
[Event \PYGZdq{}This is an 8x8 draughts game\PYGZdq{}]
[Result \PYGZdq{}1/2\PYGZhy{}1/2\PYGZdq{}]
[GameType \PYGZdq{}21\PYGZdq{}]

1.  9\PYGZhy{}14 22\PYGZhy{}17 2. 11\PYGZhy{}15 25\PYGZhy{}22 3. 15\PYGZhy{}19 \PYGZob{}Forms the Double Corner Dyke, With
 black aiming to occupy sqr 19, attacking white\PYGZsq{}s double corner.\PYGZcb{} 23x16
4. 12x19 24x15 5. 10x19 17x10 6.  6x15 21\PYGZhy{}17 7.  5\PYGZhy{}9 29\PYGZhy{}25 8.  8\PYGZhy{}12 25\PYGZhy{}21
9.  7\PYGZhy{}10 17\PYGZhy{}13 10.  1\PYGZhy{}6 \PYGZob{}It seems unwise to abondon the key back row sqr 1,
 but it is necessary to prevent 13\PYGZhy{}9..\PYGZcb{} 27\PYGZhy{}24 11.  4\PYGZhy{}8 32\PYGZhy{}27 12.  9\PYGZhy{}14
27\PYGZhy{}23 13.  3\PYGZhy{}7 23x16 14. 12x19 22\PYGZhy{}17 15.  7\PYGZhy{}11 26\PYGZhy{}23 16. 19x26 30x23
17.  8\PYGZhy{}12 24\PYGZhy{}20 18. 15\PYGZhy{}18 23\PYGZhy{}19 19. 11\PYGZhy{}15 20\PYGZhy{}16 20. 15x24 28x19 21.  2\PYGZhy{}7
31\PYGZhy{}26 22. 18\PYGZhy{}23 26\PYGZhy{}22 23. 23\PYGZhy{}27 16\PYGZhy{}11 \PYGZob{}! a really beautiful escape\PYGZcb{}
24.  7x23 22\PYGZhy{}18 *
\end{sphinxVerbatim}

\sphinxAtStartPar
An example of a live game captured with an electronic board, with clock times and two setups:

\begin{sphinxVerbatim}[commandchars=\\\{\}]
\PYG{p}{[}\PYG{n}{White} \PYG{l+s+s2}{\PYGZdq{}}\PYG{l+s+s2}{Player 1}\PYG{l+s+s2}{\PYGZdq{}}\PYG{p}{]}
\PYG{p}{[}\PYG{n}{Black} \PYG{l+s+s2}{\PYGZdq{}}\PYG{l+s+s2}{Player 2}\PYG{l+s+s2}{\PYGZdq{}}\PYG{p}{]}
\PYG{p}{[}\PYG{n}{Round} \PYG{l+s+s2}{\PYGZdq{}}\PYG{l+s+s2}{1}\PYG{l+s+s2}{\PYGZdq{}}\PYG{p}{]}
\PYG{l+m+mf}{1.}  \PYG{l+m+mi}{32}\PYG{o}{\PYGZhy{}}\PYG{l+m+mi}{28} \PYG{p}{\PYGZob{}}\PYG{p}{[}\PYG{o}{\PYGZpc{}}\PYG{n}{clock} \PYG{n}{w0}\PYG{p}{:}\PYG{l+m+mi}{00}\PYG{p}{:}\PYG{l+m+mi}{00} \PYG{n}{B0}\PYG{p}{:}\PYG{l+m+mi}{00}\PYG{p}{:}\PYG{l+m+mi}{05}\PYG{p}{]}\PYG{p}{\PYGZcb{}}
    \PYG{l+m+mi}{19}\PYG{o}{\PYGZhy{}}\PYG{l+m+mi}{23} \PYG{p}{\PYGZob{}}\PYG{p}{[}\PYG{o}{\PYGZpc{}}\PYG{n}{clock} \PYG{n}{W0}\PYG{p}{:}\PYG{l+m+mi}{00}\PYG{p}{:}\PYG{l+m+mi}{10} \PYG{n}{b0}\PYG{p}{:}\PYG{l+m+mi}{00}\PYG{p}{:}\PYG{l+m+mi}{15}\PYG{p}{]}\PYG{p}{\PYGZcb{}}
\PYG{l+m+mf}{2.}  \PYG{l+m+mi}{28}\PYG{n}{x19} \PYG{p}{\PYGZob{}}\PYG{p}{[}\PYG{o}{\PYGZpc{}}\PYG{n}{clock} \PYG{n}{w0}\PYG{p}{:}\PYG{l+m+mi}{00}\PYG{p}{:}\PYG{l+m+mi}{20} \PYG{n}{B0}\PYG{p}{:}\PYG{l+m+mi}{00}\PYG{p}{:}\PYG{l+m+mi}{25}\PYG{p}{]}\PYG{p}{\PYGZcb{}}
    \PYG{l+m+mi}{14}\PYG{n}{x23} \PYG{p}{\PYGZob{}}\PYG{p}{[}\PYG{o}{\PYGZpc{}}\PYG{n}{clock} \PYG{n}{W0}\PYG{p}{:}\PYG{l+m+mi}{00}\PYG{p}{:}\PYG{l+m+mi}{30} \PYG{n}{b0}\PYG{p}{:}\PYG{l+m+mi}{00}\PYG{p}{:}\PYG{l+m+mi}{35}\PYG{p}{]}\PYG{p}{\PYGZcb{}}
\PYG{l+m+mf}{3.}  \PYG{l+m+mi}{34}\PYG{o}{\PYGZhy{}}\PYG{l+m+mi}{29} \PYG{p}{\PYGZob{}}\PYG{p}{[}\PYG{o}{\PYGZpc{}}\PYG{n}{clock} \PYG{n}{w0}\PYG{p}{:}\PYG{l+m+mi}{00}\PYG{p}{:}\PYG{l+m+mi}{40} \PYG{n}{B0}\PYG{p}{:}\PYG{l+m+mi}{00}\PYG{p}{:}\PYG{l+m+mi}{45}\PYG{p}{]}\PYG{p}{\PYGZcb{}}
    \PYG{l+m+mi}{23}\PYG{n}{x34} \PYG{p}{\PYGZob{}}\PYG{p}{[}\PYG{o}{\PYGZpc{}}\PYG{n}{clock} \PYG{n}{W0}\PYG{p}{:}\PYG{l+m+mi}{00}\PYG{p}{:}\PYG{l+m+mi}{50} \PYG{n}{b0}\PYG{p}{:}\PYG{l+m+mi}{00}\PYG{p}{:}\PYG{l+m+mi}{55}\PYG{p}{]}\PYG{p}{\PYGZcb{}}
\PYG{l+m+mf}{4.}  \PYG{l+m+mi}{40}\PYG{n}{x29} \PYG{p}{\PYGZob{}}\PYG{p}{[}\PYG{o}{\PYGZpc{}}\PYG{n}{clock} \PYG{n}{w0}\PYG{p}{:}\PYG{l+m+mi}{01}\PYG{p}{:}\PYG{l+m+mi}{00} \PYG{n}{B0}\PYG{p}{:}\PYG{l+m+mi}{01}\PYG{p}{:}\PYG{l+m+mi}{05}\PYG{p}{]}\PYG{p}{\PYGZcb{}}
    \PYG{l+m+mi}{10}\PYG{o}{\PYGZhy{}}\PYG{l+m+mi}{14} \PYG{p}{\PYGZob{}}\PYG{p}{[}\PYG{o}{\PYGZpc{}}\PYG{n}{clock} \PYG{n}{W0}\PYG{p}{:}\PYG{l+m+mi}{01}\PYG{p}{:}\PYG{l+m+mi}{10} \PYG{n}{b0}\PYG{p}{:}\PYG{l+m+mi}{01}\PYG{p}{:}\PYG{l+m+mi}{15}\PYG{p}{]}\PYG{p}{\PYGZcb{}}
\PYG{l+m+mf}{5.}  \PYG{l+m+mi}{37}\PYG{o}{\PYGZhy{}}\PYG{l+m+mi}{32} \PYG{p}{\PYGZob{}}\PYG{p}{[}\PYG{o}{\PYGZpc{}}\PYG{n}{clock} \PYG{n}{w0}\PYG{p}{:}\PYG{l+m+mi}{01}\PYG{p}{:}\PYG{l+m+mi}{20} \PYG{n}{B0}\PYG{p}{:}\PYG{l+m+mi}{01}\PYG{p}{:}\PYG{l+m+mi}{25}\PYG{p}{]}\PYG{p}{\PYGZcb{}}
    \PYG{l+m+mi}{13}\PYG{o}{\PYGZhy{}}\PYG{l+m+mi}{19} \PYG{p}{\PYGZob{}}\PYG{p}{[}\PYG{o}{\PYGZpc{}}\PYG{n}{clock} \PYG{n}{W0}\PYG{p}{:}\PYG{l+m+mi}{01}\PYG{p}{:}\PYG{l+m+mi}{30} \PYG{n}{b0}\PYG{p}{:}\PYG{l+m+mi}{01}\PYG{p}{:}\PYG{l+m+mi}{35}\PYG{p}{]}\PYG{p}{\PYGZcb{}}
\PYG{l+m+mf}{6.}  \PYG{l+m+mi}{41}\PYG{o}{\PYGZhy{}}\PYG{l+m+mi}{37} \PYG{p}{\PYGZob{}}\PYG{p}{[}\PYG{o}{\PYGZpc{}}\PYG{n}{clock} \PYG{n}{w0}\PYG{p}{:}\PYG{l+m+mi}{01}\PYG{p}{:}\PYG{l+m+mi}{40} \PYG{n}{B0}\PYG{p}{:}\PYG{l+m+mi}{01}\PYG{p}{:}\PYG{l+m+mi}{45}\PYG{p}{]}\PYG{p}{\PYGZcb{}}
     \PYG{l+m+mi}{8}\PYG{o}{\PYGZhy{}}\PYG{l+m+mi}{13} \PYG{p}{\PYGZob{}}\PYG{p}{[}\PYG{o}{\PYGZpc{}}\PYG{n}{clock} \PYG{n}{W0}\PYG{p}{:}\PYG{l+m+mi}{01}\PYG{p}{:}\PYG{l+m+mi}{50} \PYG{n}{b0}\PYG{p}{:}\PYG{l+m+mi}{01}\PYG{p}{:}\PYG{l+m+mi}{55}\PYG{p}{]}\PYG{p}{\PYGZcb{}}
\PYG{l+m+mf}{7.}  \PYG{l+m+mi}{46}\PYG{o}{\PYGZhy{}}\PYG{l+m+mi}{41} \PYG{p}{\PYGZob{}}\PYG{p}{[}\PYG{o}{\PYGZpc{}}\PYG{n}{clock} \PYG{n}{w0}\PYG{p}{:}\PYG{l+m+mi}{02}\PYG{p}{:}\PYG{l+m+mi}{00} \PYG{n}{B0}\PYG{p}{:}\PYG{l+m+mi}{02}\PYG{p}{:}\PYG{l+m+mi}{05}\PYG{p}{]}\PYG{p}{\PYGZcb{}}
     \PYG{l+m+mi}{2}\PYG{o}{\PYGZhy{}}\PYG{l+m+mi}{8}  \PYG{p}{\PYGZob{}}\PYG{p}{[}\PYG{o}{\PYGZpc{}}\PYG{n}{clock} \PYG{n}{W0}\PYG{p}{:}\PYG{l+m+mi}{02}\PYG{p}{:}\PYG{l+m+mi}{10} \PYG{n}{b0}\PYG{p}{:}\PYG{l+m+mi}{02}\PYG{p}{:}\PYG{l+m+mi}{15}\PYG{p}{]}\PYG{p}{\PYGZcb{}}
\PYG{l+m+mf}{8.}  \PYG{l+m+mi}{45}\PYG{o}{\PYGZhy{}}\PYG{l+m+mi}{40} \PYG{p}{\PYGZob{}}\PYG{p}{[}\PYG{o}{\PYGZpc{}}\PYG{n}{clock} \PYG{n}{w0}\PYG{p}{:}\PYG{l+m+mi}{02}\PYG{p}{:}\PYG{l+m+mi}{20} \PYG{n}{B0}\PYG{p}{:}\PYG{l+m+mi}{02}\PYG{p}{:}\PYG{l+m+mi}{25}\PYG{p}{]}\PYG{p}{\PYGZcb{}}
    \PYG{l+m+mi}{17}\PYG{o}{\PYGZhy{}}\PYG{l+m+mi}{21} \PYG{p}{\PYGZob{}}\PYG{p}{[}\PYG{o}{\PYGZpc{}}\PYG{n}{clock} \PYG{n}{W0}\PYG{p}{:}\PYG{l+m+mi}{02}\PYG{p}{:}\PYG{l+m+mi}{30} \PYG{n}{b0}\PYG{p}{:}\PYG{l+m+mi}{02}\PYG{p}{:}\PYG{l+m+mi}{35}\PYG{p}{]}\PYG{p}{\PYGZcb{}}
\PYG{l+m+mf}{9.}  \PYG{l+m+mi}{31}\PYG{o}{\PYGZhy{}}\PYG{l+m+mi}{26} \PYG{p}{\PYGZob{}}\PYG{p}{[}\PYG{o}{\PYGZpc{}}\PYG{n}{clock} \PYG{n}{w0}\PYG{p}{:}\PYG{l+m+mi}{02}\PYG{p}{:}\PYG{l+m+mi}{40} \PYG{n}{B0}\PYG{p}{:}\PYG{l+m+mi}{02}\PYG{p}{:}\PYG{l+m+mi}{45}\PYG{p}{]}\PYG{p}{\PYGZcb{}}
    \PYG{l+m+mi}{19}\PYG{o}{\PYGZhy{}}\PYG{l+m+mi}{23} \PYG{p}{\PYGZob{}}\PYG{p}{[}\PYG{o}{\PYGZpc{}}\PYG{n}{clock} \PYG{n}{W0}\PYG{p}{:}\PYG{l+m+mi}{02}\PYG{p}{:}\PYG{l+m+mi}{50} \PYG{n}{b0}\PYG{p}{:}\PYG{l+m+mi}{02}\PYG{p}{:}\PYG{l+m+mi}{55}\PYG{p}{]}\PYG{p}{\PYGZcb{}}
\PYG{l+m+mf}{10.} \PYG{l+m+mi}{26}\PYG{n}{x17} \PYG{p}{\PYGZob{}}\PYG{p}{[}\PYG{o}{\PYGZpc{}}\PYG{n}{clock} \PYG{n}{w0}\PYG{p}{:}\PYG{l+m+mi}{03}\PYG{p}{:}\PYG{l+m+mi}{00} \PYG{n}{B0}\PYG{p}{:}\PYG{l+m+mi}{03}\PYG{p}{:}\PYG{l+m+mi}{05}\PYG{p}{]}\PYG{p}{\PYGZcb{}}
    \PYG{l+m+mi}{23}\PYG{n}{x45} \PYG{p}{\PYGZob{}}\PYG{p}{[}\PYG{o}{\PYGZpc{}}\PYG{n}{clock} \PYG{n}{W0}\PYG{p}{:}\PYG{l+m+mi}{03}\PYG{p}{:}\PYG{l+m+mi}{10} \PYG{n}{b0}\PYG{p}{:}\PYG{l+m+mi}{03}\PYG{p}{:}\PYG{l+m+mi}{15}\PYG{p}{]}\PYG{p}{\PYGZcb{}}
\PYG{l+m+mf}{11.} \PYG{l+m+mi}{36}\PYG{o}{\PYGZhy{}}\PYG{l+m+mi}{31} \PYG{p}{\PYGZob{}}\PYG{p}{[}\PYG{o}{\PYGZpc{}}\PYG{n}{clock} \PYG{n}{w0}\PYG{p}{:}\PYG{l+m+mi}{03}\PYG{p}{:}\PYG{l+m+mi}{20} \PYG{n}{B0}\PYG{p}{:}\PYG{l+m+mi}{03}\PYG{p}{:}\PYG{l+m+mi}{25}\PYG{p}{]}\PYG{p}{\PYGZcb{}}
    \PYG{l+m+mi}{12}\PYG{n}{x21} \PYG{p}{\PYGZob{}}\PYG{p}{[}\PYG{o}{\PYGZpc{}}\PYG{n}{clock} \PYG{n}{W0}\PYG{p}{:}\PYG{l+m+mi}{03}\PYG{p}{:}\PYG{l+m+mi}{30} \PYG{n}{b0}\PYG{p}{:}\PYG{l+m+mi}{03}\PYG{p}{:}\PYG{l+m+mi}{35}\PYG{p}{]}\PYG{p}{\PYGZcb{}}
\PYG{l+m+mf}{12.} \PYG{l+m+mi}{31}\PYG{o}{\PYGZhy{}}\PYG{l+m+mi}{26} \PYG{p}{\PYGZob{}}\PYG{p}{[}\PYG{o}{\PYGZpc{}}\PYG{n}{clock} \PYG{n}{w0}\PYG{p}{:}\PYG{l+m+mi}{03}\PYG{p}{:}\PYG{l+m+mi}{40} \PYG{n}{B0}\PYG{p}{:}\PYG{l+m+mi}{03}\PYG{p}{:}\PYG{l+m+mi}{45}\PYG{p}{]}\PYG{p}{\PYGZcb{}}
     \PYG{l+m+mi}{7}\PYG{o}{\PYGZhy{}}\PYG{l+m+mi}{12} \PYG{p}{\PYGZob{}}\PYG{p}{[}\PYG{o}{\PYGZpc{}}\PYG{n}{clock} \PYG{n}{W0}\PYG{p}{:}\PYG{l+m+mi}{03}\PYG{p}{:}\PYG{l+m+mi}{50} \PYG{n}{b0}\PYG{p}{:}\PYG{l+m+mi}{03}\PYG{p}{:}\PYG{l+m+mi}{55}\PYG{p}{]}\PYG{p}{\PYGZcb{}}
\PYG{l+m+mf}{13.} \PYG{l+m+mi}{26}\PYG{n}{x17} \PYG{p}{\PYGZob{}}\PYG{p}{[}\PYG{o}{\PYGZpc{}}\PYG{n}{clock} \PYG{n}{w0}\PYG{p}{:}\PYG{l+m+mi}{04}\PYG{p}{:}\PYG{l+m+mi}{00} \PYG{n}{B0}\PYG{p}{:}\PYG{l+m+mi}{04}\PYG{p}{:}\PYG{l+m+mi}{05}\PYG{p}{]}\PYG{p}{\PYGZcb{}}
    \PYG{l+m+mi}{12}\PYG{n}{x21} \PYG{p}{\PYGZob{}}\PYG{p}{[}\PYG{o}{\PYGZpc{}}\PYG{n}{clock} \PYG{n}{W0}\PYG{p}{:}\PYG{l+m+mi}{04}\PYG{p}{:}\PYG{l+m+mi}{10} \PYG{n}{b0}\PYG{p}{:}\PYG{l+m+mi}{04}\PYG{p}{:}\PYG{l+m+mi}{15}\PYG{p}{]}\PYG{p}{\PYGZcb{}}
\PYG{l+m+mf}{14.} \PYG{l+m+mi}{33}\PYG{o}{\PYGZhy{}}\PYG{l+m+mi}{29} \PYG{p}{\PYGZob{}}\PYG{p}{[}\PYG{o}{\PYGZpc{}}\PYG{n}{clock} \PYG{n}{w0}\PYG{p}{:}\PYG{l+m+mi}{04}\PYG{p}{:}\PYG{l+m+mi}{20} \PYG{n}{B0}\PYG{p}{:}\PYG{l+m+mi}{04}\PYG{p}{:}\PYG{l+m+mi}{25}\PYG{p}{]}\PYG{p}{\PYGZcb{}}
    \PYG{l+m+mi}{21}\PYG{o}{\PYGZhy{}}\PYG{l+m+mi}{26} \PYG{p}{\PYGZob{}}\PYG{p}{[}\PYG{o}{\PYGZpc{}}\PYG{n}{clock} \PYG{n}{W0}\PYG{p}{:}\PYG{l+m+mi}{04}\PYG{p}{:}\PYG{l+m+mi}{30} \PYG{n}{b0}\PYG{p}{:}\PYG{l+m+mi}{04}\PYG{p}{:}\PYG{l+m+mi}{35}\PYG{p}{]}\PYG{p}{\PYGZcb{}}
\PYG{l+m+mf}{15.} \PYG{l+m+mi}{41}\PYG{o}{\PYGZhy{}}\PYG{l+m+mi}{36} \PYG{p}{\PYGZob{}}\PYG{p}{[}\PYG{o}{\PYGZpc{}}\PYG{n}{clock} \PYG{n}{w0}\PYG{p}{:}\PYG{l+m+mi}{04}\PYG{p}{:}\PYG{l+m+mi}{40} \PYG{n}{B0}\PYG{p}{:}\PYG{l+m+mi}{04}\PYG{p}{:}\PYG{l+m+mi}{45}\PYG{p}{]}\PYG{p}{\PYGZcb{}}
    \PYG{l+m+mi}{14}\PYG{o}{\PYGZhy{}}\PYG{l+m+mi}{19} \PYG{p}{\PYGZob{}}\PYG{p}{[}\PYG{o}{\PYGZpc{}}\PYG{n}{clock} \PYG{n}{W0}\PYG{p}{:}\PYG{l+m+mi}{04}\PYG{p}{:}\PYG{l+m+mi}{50} \PYG{n}{b0}\PYG{p}{:}\PYG{l+m+mi}{04}\PYG{p}{:}\PYG{l+m+mi}{55}\PYG{p}{]}\PYG{p}{\PYGZcb{}}
\PYG{l+m+mf}{16.} \PYG{l+m+mi}{39}\PYG{o}{\PYGZhy{}}\PYG{l+m+mi}{33} \PYG{p}{\PYGZob{}}\PYG{p}{[}\PYG{o}{\PYGZpc{}}\PYG{n}{clock} \PYG{n}{w0}\PYG{p}{:}\PYG{l+m+mi}{05}\PYG{p}{:}\PYG{l+m+mi}{00} \PYG{n}{B0}\PYG{p}{:}\PYG{l+m+mi}{05}\PYG{p}{:}\PYG{l+m+mi}{05}\PYG{p}{]}\PYG{p}{\PYGZcb{}}
    \PYG{l+m+mi}{20}\PYG{o}{\PYGZhy{}}\PYG{l+m+mi}{24} \PYG{p}{\PYGZob{}}\PYG{p}{[}\PYG{o}{\PYGZpc{}}\PYG{n}{clock} \PYG{n}{W0}\PYG{p}{:}\PYG{l+m+mi}{05}\PYG{p}{:}\PYG{l+m+mi}{10} \PYG{n}{b0}\PYG{p}{:}\PYG{l+m+mi}{05}\PYG{p}{:}\PYG{l+m+mi}{15}\PYG{p}{]}\PYG{p}{\PYGZcb{}}
\PYG{l+m+mf}{17.} \PYG{l+m+mi}{29}\PYG{n}{x20} \PYG{p}{\PYGZob{}}\PYG{p}{[}\PYG{o}{\PYGZpc{}}\PYG{n}{clock} \PYG{n}{w0}\PYG{p}{:}\PYG{l+m+mi}{05}\PYG{p}{:}\PYG{l+m+mi}{20} \PYG{n}{B0}\PYG{p}{:}\PYG{l+m+mi}{05}\PYG{p}{:}\PYG{l+m+mi}{25}\PYG{p}{]}\PYG{p}{\PYGZcb{}}
    \PYG{l+m+mi}{15}\PYG{n}{x24} \PYG{p}{\PYGZob{}}\PYG{p}{[}\PYG{o}{\PYGZpc{}}\PYG{n}{clock} \PYG{n}{W0}\PYG{p}{:}\PYG{l+m+mi}{05}\PYG{p}{:}\PYG{l+m+mi}{30} \PYG{n}{b0}\PYG{p}{:}\PYG{l+m+mi}{05}\PYG{p}{:}\PYG{l+m+mi}{35}\PYG{p}{]}\PYG{p}{\PYGZcb{}}
\PYG{o}{/}\PYG{n}{FEN} \PYG{l+s+s2}{\PYGZdq{}}\PYG{l+s+s2}{W:W32,33,35,36,37,38,42,43,44,47,48,49,50:B1,3,4,5,6,8,9,11,13,16,18,26,30,45}\PYG{l+s+s2}{\PYGZdq{}}\PYG{o}{/} \PYG{p}{\PYGZob{}}\PYG{p}{[}\PYG{o}{\PYGZpc{}}\PYG{n}{clock} \PYG{n}{W0}\PYG{p}{:}\PYG{l+m+mi}{05}\PYG{p}{:}\PYG{l+m+mi}{40} \PYG{n}{b0}\PYG{p}{:}\PYG{l+m+mi}{05}\PYG{p}{:}\PYG{l+m+mi}{45}\PYG{p}{]}\PYG{p}{\PYGZcb{}}
\PYG{l+m+mf}{19.} \PYG{l+m+mi}{35}\PYG{n}{x24} \PYG{p}{\PYGZob{}}\PYG{p}{[}\PYG{o}{\PYGZpc{}}\PYG{n}{clock} \PYG{n}{w0}\PYG{p}{:}\PYG{l+m+mi}{05}\PYG{p}{:}\PYG{l+m+mi}{50} \PYG{n}{B0}\PYG{p}{:}\PYG{l+m+mi}{05}\PYG{p}{:}\PYG{l+m+mi}{55}\PYG{p}{]}\PYG{p}{\PYGZcb{}}
\PYG{o}{/}\PYG{n}{FEN} \PYG{l+s+s2}{\PYGZdq{}}\PYG{l+s+s2}{W:W32,33,36,37,38,42,43,44,47,48,49,50:B1,3,4,5,6,8,9,11,13,16,18,26,30,45}\PYG{l+s+s2}{\PYGZdq{}}\PYG{o}{/} \PYG{p}{\PYGZob{}}\PYG{p}{[}\PYG{o}{\PYGZpc{}}\PYG{n}{clock} \PYG{n}{W0}\PYG{p}{:}\PYG{l+m+mi}{06}\PYG{p}{:}\PYG{l+m+mi}{00} \PYG{n}{b0}\PYG{p}{:}\PYG{l+m+mi}{06}\PYG{p}{:}\PYG{l+m+mi}{05}\PYG{p}{]}\PYG{p}{\PYGZcb{}}
\PYG{l+m+mf}{21.} \PYG{l+m+mi}{32}\PYG{o}{\PYGZhy{}}\PYG{l+m+mi}{28} \PYG{p}{\PYGZob{}}\PYG{p}{[}\PYG{o}{\PYGZpc{}}\PYG{n}{clock} \PYG{n}{w0}\PYG{p}{:}\PYG{l+m+mi}{06}\PYG{p}{:}\PYG{l+m+mi}{10} \PYG{n}{B0}\PYG{p}{:}\PYG{l+m+mi}{06}\PYG{p}{:}\PYG{l+m+mi}{15}\PYG{p}{]}\PYG{p}{\PYGZcb{}}
    \PYG{l+m+mi}{30}\PYG{o}{\PYGZhy{}}\PYG{l+m+mi}{35} \PYG{p}{\PYGZob{}}\PYG{p}{[}\PYG{o}{\PYGZpc{}}\PYG{n}{clock} \PYG{n}{W0}\PYG{p}{:}\PYG{l+m+mi}{06}\PYG{p}{:}\PYG{l+m+mi}{20} \PYG{n}{b0}\PYG{p}{:}\PYG{l+m+mi}{06}\PYG{p}{:}\PYG{l+m+mi}{25}\PYG{p}{]}\PYG{p}{\PYGZcb{}}
\PYG{l+m+mf}{22.} \PYG{l+m+mi}{43}\PYG{o}{\PYGZhy{}}\PYG{l+m+mi}{39} \PYG{p}{\PYGZob{}}\PYG{p}{[}\PYG{o}{\PYGZpc{}}\PYG{n}{clock} \PYG{n}{w0}\PYG{p}{:}\PYG{l+m+mi}{06}\PYG{p}{:}\PYG{l+m+mi}{30} \PYG{n}{B0}\PYG{p}{:}\PYG{l+m+mi}{06}\PYG{p}{:}\PYG{l+m+mi}{35}\PYG{p}{]}\PYG{p}{\PYGZcb{}}
     \PYG{l+m+mi}{1}\PYG{o}{\PYGZhy{}}\PYG{l+m+mi}{7}  \PYG{p}{\PYGZob{}}\PYG{p}{[}\PYG{o}{\PYGZpc{}}\PYG{n}{clock} \PYG{n}{W0}\PYG{p}{:}\PYG{l+m+mi}{06}\PYG{p}{:}\PYG{l+m+mi}{40} \PYG{n}{b0}\PYG{p}{:}\PYG{l+m+mi}{06}\PYG{p}{:}\PYG{l+m+mi}{45}\PYG{p}{]}\PYG{p}{\PYGZcb{}}
\PYG{l+m+mf}{23.} \PYG{l+m+mi}{37}\PYG{o}{\PYGZhy{}}\PYG{l+m+mi}{32} \PYG{p}{\PYGZob{}}\PYG{p}{[}\PYG{o}{\PYGZpc{}}\PYG{n}{clock} \PYG{n}{w0}\PYG{p}{:}\PYG{l+m+mi}{06}\PYG{p}{:}\PYG{l+m+mi}{50} \PYG{n}{B0}\PYG{p}{:}\PYG{l+m+mi}{06}\PYG{p}{:}\PYG{l+m+mi}{55}\PYG{p}{]}\PYG{p}{\PYGZcb{}}
     \PYG{l+m+mi}{7}\PYG{o}{\PYGZhy{}}\PYG{l+m+mi}{12} \PYG{p}{\PYGZob{}}\PYG{p}{[}\PYG{o}{\PYGZpc{}}\PYG{n}{clock} \PYG{n}{W0}\PYG{p}{:}\PYG{l+m+mi}{07}\PYG{p}{:}\PYG{l+m+mi}{00} \PYG{n}{b0}\PYG{p}{:}\PYG{l+m+mi}{07}\PYG{p}{:}\PYG{l+m+mi}{05}\PYG{p}{]}\PYG{p}{\PYGZcb{}}
\PYG{l+m+mf}{24.} \PYG{l+m+mi}{49}\PYG{o}{\PYGZhy{}}\PYG{l+m+mi}{43} \PYG{p}{\PYGZob{}}\PYG{p}{[}\PYG{o}{\PYGZpc{}}\PYG{n}{clock} \PYG{n}{w0}\PYG{p}{:}\PYG{l+m+mi}{07}\PYG{p}{:}\PYG{l+m+mi}{10} \PYG{n}{B0}\PYG{p}{:}\PYG{l+m+mi}{07}\PYG{p}{:}\PYG{l+m+mi}{15}\PYG{p}{]}\PYG{p}{\PYGZcb{}}
    \PYG{l+m+mi}{18}\PYG{o}{\PYGZhy{}}\PYG{l+m+mi}{22} \PYG{p}{\PYGZob{}}\PYG{p}{[}\PYG{o}{\PYGZpc{}}\PYG{n}{clock} \PYG{n}{W0}\PYG{p}{:}\PYG{l+m+mi}{07}\PYG{p}{:}\PYG{l+m+mi}{20} \PYG{n}{b0}\PYG{p}{:}\PYG{l+m+mi}{07}\PYG{p}{:}\PYG{l+m+mi}{25}\PYG{p}{]}\PYG{p}{\PYGZcb{}}
\PYG{l+m+mf}{25.} \PYG{l+m+mi}{28}\PYG{n}{x17} \PYG{p}{\PYGZob{}}\PYG{p}{[}\PYG{o}{\PYGZpc{}}\PYG{n}{clock} \PYG{n}{w0}\PYG{p}{:}\PYG{l+m+mi}{07}\PYG{p}{:}\PYG{l+m+mi}{30} \PYG{n}{B0}\PYG{p}{:}\PYG{l+m+mi}{07}\PYG{p}{:}\PYG{l+m+mi}{35}\PYG{p}{]}\PYG{p}{\PYGZcb{}}
    \PYG{l+m+mi}{12}\PYG{n}{x21} \PYG{p}{\PYGZob{}}\PYG{p}{[}\PYG{o}{\PYGZpc{}}\PYG{n}{clock} \PYG{n}{W0}\PYG{p}{:}\PYG{l+m+mi}{07}\PYG{p}{:}\PYG{l+m+mi}{40} \PYG{n}{b0}\PYG{p}{:}\PYG{l+m+mi}{07}\PYG{p}{:}\PYG{l+m+mi}{45}\PYG{p}{]}\PYG{p}{\PYGZcb{}}
\end{sphinxVerbatim}

\sphinxstepscope


\chapter{PDN parsing issues}
\label{\detokenize{issues:pdn-parsing-issues}}\label{\detokenize{issues:issues-section}}\label{\detokenize{issues::doc}}
\sphinxAtStartPar
This section gives an overview of some issues with existing PDN definitions. In particular
those issues that are important when writing a PDN parser.


\section{Game Separator (1)}
\label{\detokenize{issues:game-separator-1}}
\sphinxAtStartPar
The \sphinxcode{\sphinxupquote{*}} symbol is both used as a game terminator/separator and as a move strength
indicator to denote a forced move. This introduces a nasty ambiguity.
For example the string \sphinxcode{\sphinxupquote{1\sphinxhyphen{}6* 32\sphinxhyphen{}28}} can be interpreted as one game containing two moves,
or as two games separated by a \sphinxcode{\sphinxupquote{*}}. Since \sphinxcode{\sphinxupquote{*}} is commonly used in draughts publications
to denote forced moves, the preferred solution would be to disallow \sphinxcode{\sphinxupquote{*}} as a game separator,
and to use a different symbol like \sphinxcode{\sphinxupquote{\#}}. However, this would completely destroy backward
compatibility. A less intrusive solution is to disallow \sphinxcode{\sphinxupquote{*}} as a move strength indicator.
Note that there is an alternative available in the form of the \sphinxcode{\sphinxupquote{\$7}} numeric annotation
glyph. Yet another solution is to demand that there can be no space between a move and it’s
corresponding move strength. Then a move and it’s corresponding move strength can be
defined as one token.


\section{Game Separator (2)}
\label{\detokenize{issues:game-separator-2}}
\sphinxAtStartPar
It is common practice to terminate games with their result. In PGN this is no problem,
since the chess results differ substantially from chess moves. But in draughts some
results like \sphinxcode{\sphinxupquote{1\sphinxhyphen{}0}} and \sphinxcode{\sphinxupquote{1\sphinxhyphen{}1}} are very similar to normal draughts moves. This complicates
parsing. For example, if the result \sphinxcode{\sphinxupquote{1\sphinxhyphen{}1}} is defined as a token, then parsers may easily
get confused by a move like \sphinxcode{\sphinxupquote{1\sphinxhyphen{}18}}. Several parsers insist on parsing this as
\sphinxcode{\sphinxupquote{1\sphinxhyphen{}1}} followed by an \sphinxcode{\sphinxupquote{8}}. This problem is likely to occur when a move is split up into
separate tokens.

\sphinxAtStartPar
Since the result of a game can already be specified in PDN using the \sphinxcode{\sphinxupquote{Result}} tag, there
is no need to use a game result as a game separator. It can even be considered as bad style
to have two different ways to specify the result of a game. It seems therefore logical to
forbid using the result of a game as a game terminator (or separator).


\section{Capture Separator}
\label{\detokenize{issues:capture-separator}}
\sphinxAtStartPar
The squares of a capture are separated using the symbol \sphinxcode{\sphinxupquote{x}}, for example in the
move \sphinxcode{\sphinxupquote{32x23}}. If one defines a capture as a production
\begin{quote}
\begin{quote}

\sphinxAtStartPar
CaptureMove = Square “x” Square
\end{quote}
\end{quote}

\sphinxAtStartPar
then there can easily be conflicts with identifier tokens. Tokenizers are often greedy,
which means that they can insist on parsing \sphinxcode{\sphinxupquote{x23}} as an identifier token, instead
of a capture separator \sphinxcode{\sphinxupquote{x}} followed by a square \sphinxcode{\sphinxupquote{23}}. Some parsers offer solutions to
this type of problem, but not all of them. Note that this problem can be avoided by defining
a move as a single token.


\section{Move token}
\label{\detokenize{issues:move-token}}
\sphinxAtStartPar
It is an important question whether a move should be defined as a single token (by means of
a regular expression), or as a production consisting of multiple elements. A production has
the benefit that the structure of a move can be represented more clearly. But as explained
above, then a more powerful parser is needed. If a move is defined as a token, then a
simple LL(1) parser is enough to parse PDN.


\section{Move strength}
\label{\detokenize{issues:move-strength}}
\sphinxAtStartPar
In draughts publications a move strength can be wrapped in parentheses, like in \sphinxcode{\sphinxupquote{31\sphinxhyphen{}27(?)}}.
Parentheses are also used to define variations in an analysis, for example
\sphinxcode{\sphinxupquote{1.32\sphinxhyphen{}28 18\sphinxhyphen{}23 2.38\sphinxhyphen{}32 ( 2.37\sphinxhyphen{}32? 23\sphinxhyphen{}29! ) 12\sphinxhyphen{}18}}. This introduces an ambiguity, but in
most parsers this can be resolved by defining a move strength as a single token.

\sphinxstepscope


\chapter{FEN tag}
\label{\detokenize{fen:fen-tag}}\label{\detokenize{fen:fen-section}}\label{\detokenize{fen::doc}}
\sphinxAtStartPar
The Forsyth\sphinxhyphen{}Edwards Notation or shortly FEN tag defines a position on the board.
In the PDN 2.0 draft standard the following syntax is proposed:

\sphinxAtStartPar
\sphinxcode{\sphinxupquote{{[}TURN{]}:{[}COLOUR1{]}{[}{[}K{]}{[}SQUARE\_NUM{]}{[},{]}...{]}:{[}COLOUR2{]}{[}{[}K{]}{[}SQUARE\_NUM{]}{[},{]}...{]}}}

\sphinxAtStartPar
We extend this syntax with ranges of numeric fields, such that the start
position of international draughts can be defined as \sphinxcode{\sphinxupquote{{[}FEN "W:W31\sphinxhyphen{}50:B1\sphinxhyphen{}20"{]}}}.

\sphinxAtStartPar
Another extension is that we allow \sphinxcode{\sphinxupquote{?}} to denote an unknown color.

\sphinxAtStartPar
The following grammar describes the formal syntax of the value of a FEN tag.

\begin{sphinxVerbatim}[commandchars=\\\{\}]
// Productions
Fen                        : COLOR (NumericSquares | AlphaNumericSquares) DOT?
NumericSquares             : (COLON COLOR NumericSquareSequence)+
NumericSquareSequence      : NumericSquareRange (COMMA NumericSquareRange)*
NumericSquareRange         : KING? NUMSQUARE (HYPHEN NUMSQUARE)?
AlphaNumericSquares        : (COLON COLOR AlphaNumericSquareSequence)+
AlphaNumericSquareSequence : KING? ALPHASQUARE (COMMA KING? ALPHASQUARE)*

// Tokens
COLOR                      : \PYGZdq{}[WB?]\PYGZdq{}
KING                       : \PYGZdq{}K\PYGZdq{}
ALPHASQUARE                : \PYGZdq{}[a\PYGZhy{}h][1\PYGZhy{}8]\PYGZdq{}
NUMSQUARE                  : \PYGZdq{}([1\PYGZhy{}9][0\PYGZhy{}9]*)|(0[1\PYGZhy{}9][0\PYGZhy{}9]*)|0\PYGZdq{}
HYPHEN                     : \PYGZdq{}\PYGZbs{}\PYGZhy{}\PYGZdq{}
COMMA                      : \PYGZdq{}\PYGZbs{},\PYGZdq{}
COLON                      : \PYGZdq{}\PYGZbs{}:\PYGZdq{}
DOT                        : \PYGZdq{}\PYGZbs{}.\PYGZdq{}
\end{sphinxVerbatim}

\begin{sphinxadmonition}{important}{Important:}
\sphinxAtStartPar
Some corrections have been made.
\begin{itemize}
\item {} 
\sphinxAtStartPar
The \sphinxcode{\sphinxupquote{COMMA}} in \sphinxcode{\sphinxupquote{NumericSquareSequence}} and \sphinxcode{\sphinxupquote{AlphaNumericSquareSequence}} is mandatory.

\item {} 
\sphinxAtStartPar
A missing \sphinxcode{\sphinxupquote{KING?}} in \sphinxcode{\sphinxupquote{AlphaNumericSquareSequence}} has been added.

\item {} 
\sphinxAtStartPar
\sphinxcode{\sphinxupquote{NUMSQUARE}} may contain more than two digits (it is up to the implementer to do range checks).

\end{itemize}
\end{sphinxadmonition}


\section{Restrictions}
\label{\detokenize{fen:restrictions}}
\sphinxAtStartPar
The following restrictions apply when writing a FEN tag:
\begin{itemize}
\item {} 
\sphinxAtStartPar
No embedded spaces are allowed inside the value of a FEN tag

\item {} 
\sphinxAtStartPar
No dot (‘.’) is allowed at the end of the value of a FEN tag

\end{itemize}


\section{Examples}
\label{\detokenize{fen:examples}}
\begin{sphinxVerbatim}[commandchars=\\\{\}]
\PYG{p}{[}\PYG{n}{FEN} \PYG{l+s+s2}{\PYGZdq{}}\PYG{l+s+s2}{B:W18,24,27,28,K10,K15:B12,16,20,K22,K25,K29}\PYG{l+s+s2}{\PYGZdq{}}\PYG{p}{]}
\PYG{p}{[}\PYG{n}{FEN} \PYG{l+s+s2}{\PYGZdq{}}\PYG{l+s+s2}{B:W18,19,21,23,24,26,29,30,31,32:B1,2,3,4,6,7,9,10,11,12}\PYG{l+s+s2}{\PYGZdq{}}\PYG{p}{]}
\PYG{p}{[}\PYG{n}{FEN} \PYG{l+s+s2}{\PYGZdq{}}\PYG{l+s+s2}{W:W31\PYGZhy{}50:B1\PYGZhy{}20}\PYG{l+s+s2}{\PYGZdq{}}\PYG{p}{]}
\end{sphinxVerbatim}


\section{Extensions}
\label{\detokenize{fen:extensions}}
\sphinxAtStartPar
The above grammar does not allow positions with no pieces for one of the
players. It should therefore be extended to accept empty ranges of pieces.

\sphinxstepscope


\chapter{GameType tag}
\label{\detokenize{gametype:gametype-tag}}\label{\detokenize{gametype:gametype-section}}\label{\detokenize{gametype::doc}}
\sphinxAtStartPar
The GameType tag defines the type of the draughts game, including the size of the board
and the preferred notation. It has been previously defined on \sphinxcite{introduction:grimminck}.

\sphinxAtStartPar
The GameType tag has the following syntax:

\sphinxAtStartPar
\sphinxcode{\sphinxupquote{GameType "Type\sphinxhyphen{}number {[},Start colour (W/B),Board width, Board height, Notation {[},Invert\sphinxhyphen{}flag{]}{]}"}}

\sphinxAtStartPar
A regular expression for the GameType tag is

\begin{sphinxVerbatim}[commandchars=\\\{\}]
\PYG{l+s+s2}{\PYGZdq{}}\PYG{l+s+s2}{[0\PYGZhy{}9]+(,[WB],[0\PYGZhy{}9]+,[0\PYGZhy{}9]+,[ANS][0123](,[01])?)?}\PYG{l+s+s2}{\PYGZdq{}}
\end{sphinxVerbatim}

\sphinxAtStartPar
The game type is a number followed by some optional attributes. Several numbers are
predefined, as given by the following table. A test page for the GameType tag can be
found at \sphinxcode{\sphinxupquote{gametype.html}}.


\begin{savenotes}\sphinxattablestart
\sphinxthistablewithglobalstyle
\centering
\begin{tabulary}{\linewidth}[t]{TTTTT}
\sphinxtoprule
\sphinxstyletheadfamily 
\sphinxAtStartPar
Type
&\sphinxstyletheadfamily 
\sphinxAtStartPar
Game
&\sphinxstyletheadfamily 
\sphinxAtStartPar
Full details
&\sphinxstyletheadfamily 
\sphinxAtStartPar
Result type
&\sphinxstyletheadfamily 
\sphinxAtStartPar
Capture Separator
\\
\sphinxmidrule
\sphinxtableatstartofbodyhook
\sphinxAtStartPar
0
&
\sphinxAtStartPar
Chess
&&&\\
\sphinxhline
\sphinxAtStartPar
1
&
\sphinxAtStartPar
Chinese chess
&&&\\
\sphinxhline
\sphinxAtStartPar
2\sphinxhyphen{}19
&
\sphinxAtStartPar
Future chess expansion
&&&\\
\sphinxhline
\sphinxAtStartPar
20
&
\sphinxAtStartPar
10x10 International draughts
&
\sphinxAtStartPar
{[}GameType “20,W,10,10,N2,0”{]}
&
\sphinxAtStartPar
International
&
\sphinxAtStartPar
x
\\
\sphinxhline
\sphinxAtStartPar
21
&
\sphinxAtStartPar
English draughts
&
\sphinxAtStartPar
{[}GameType “21,B,8,8,N1,0”{]}
&
\sphinxAtStartPar
Default
&
\sphinxAtStartPar
x
\\
\sphinxhline
\sphinxAtStartPar
22
&
\sphinxAtStartPar
Italian draughts
&
\sphinxAtStartPar
{[}GameType “22,W,8,8,N2,1”{]}
&
\sphinxAtStartPar
Default
&
\sphinxAtStartPar
x
\\
\sphinxhline
\sphinxAtStartPar
23
&
\sphinxAtStartPar
American pool checkers
&
\sphinxAtStartPar
{[}GameType “23,B,8,8,N1,0”{]}
&
\sphinxAtStartPar
Default
&
\sphinxAtStartPar
x
\\
\sphinxhline&
\sphinxAtStartPar
Pool checkers (unified) \sphinxstylestrong{*)}
&
\sphinxAtStartPar
{[}GameType “23,W,8,8,A0,0”{]}
&
\sphinxAtStartPar
Default
&
\sphinxAtStartPar
x
\\
\sphinxhline&
\sphinxAtStartPar
Zimbabwean pool checkers \sphinxstylestrong{*)}
&
\sphinxAtStartPar
{[}GameType “23,W,8,8,A0,0”{]}
&
\sphinxAtStartPar
Default
&
\sphinxAtStartPar
x
\\
\sphinxhline&
\sphinxAtStartPar
Jamaican draughts \sphinxstylestrong{*)}
&
\sphinxAtStartPar
{[}GameType “23,W,8,8,A1,1”{]}
&
\sphinxAtStartPar
Default
&
\sphinxAtStartPar
x
\\
\sphinxhline
\sphinxAtStartPar
24
&
\sphinxAtStartPar
Spanish draughts
&
\sphinxAtStartPar
{[}GameType “24,W,8,8,N1,1”{]}
&
\sphinxAtStartPar
Default
&
\sphinxAtStartPar
x
\\
\sphinxhline
\sphinxAtStartPar
25
&
\sphinxAtStartPar
Russian draughts
&
\sphinxAtStartPar
{[}GameType “25,W,8,8,A0,0”{]}
&
\sphinxAtStartPar
Default
&
\sphinxAtStartPar
:
\\
\sphinxhline
\sphinxAtStartPar
26
&
\sphinxAtStartPar
Brazilian draughts
&
\sphinxAtStartPar
{[}GameType “26,W,8,8,A0,0”{]}
&
\sphinxAtStartPar
Default
&
\sphinxAtStartPar
x
\\
\sphinxhline
\sphinxAtStartPar
27
&
\sphinxAtStartPar
Canadian draughts
&
\sphinxAtStartPar
{[}GameType “27,W,12,12,N2,0”{]}
&
\sphinxAtStartPar
International
&
\sphinxAtStartPar
x
\\
\sphinxhline
\sphinxAtStartPar
28
&
\sphinxAtStartPar
Portuguese draughts
&
\sphinxAtStartPar
{[}GameType “28,W,8,8,N1,1”{]}
&
\sphinxAtStartPar
Default
&
\sphinxAtStartPar
x
\\
\sphinxhline
\sphinxAtStartPar
29
&
\sphinxAtStartPar
Czech draughts
&
\sphinxAtStartPar
{[}GameType “29,W,8,8,A0,0”{]}
&
\sphinxAtStartPar
Default
&
\sphinxAtStartPar
x
\\
\sphinxhline
\sphinxAtStartPar
30
&
\sphinxAtStartPar
Turkish draughts
&
\sphinxAtStartPar
{[}GameType “30,W,8,8,A0,0”{]}
&
\sphinxAtStartPar
Default
&
\sphinxAtStartPar
x
\\
\sphinxhline
\sphinxAtStartPar
31
&
\sphinxAtStartPar
Thai draughts
&
\sphinxAtStartPar
{[}GameType “31,B,8,8,N2,0”{]}
&
\sphinxAtStartPar
Default
&
\sphinxAtStartPar
\sphinxhyphen{}
\\
\sphinxhline
\sphinxAtStartPar
40
&
\sphinxAtStartPar
Frisian draughts
&
\sphinxAtStartPar
{[}GameType “40,W,10,10,N2,0”{]}
&
\sphinxAtStartPar
Default
&
\sphinxAtStartPar
x
\\
\sphinxhline
\sphinxAtStartPar
41
&
\sphinxAtStartPar
Spantsiretti draughts
&
\sphinxAtStartPar
{[}GameType “41,W,10,8,A0,0”{]}
&
\sphinxAtStartPar
Default
&
\sphinxAtStartPar
:
\\
\sphinxhline
\sphinxAtStartPar
42\sphinxhyphen{}49
&
\sphinxAtStartPar
Future draughts expansion
&&&\\
\sphinxhline
\sphinxAtStartPar
50
&
\sphinxAtStartPar
Othello
&&&\\
\sphinxhline
\sphinxAtStartPar
51..
&
\sphinxAtStartPar
Future expansion
&&&\\
\sphinxbottomrule
\end{tabulary}
\sphinxtableafterendhook\par
\sphinxattableend\end{savenotes}

\sphinxAtStartPar
\sphinxstylestrong{*)} Note that Pool checkers and Jamaican draughts have
been added later. They have the same rules as American pool checkers, but they use algebraic
notation, and white moves first. Moreover, in Jamaican draughts the direction of the
numbering is vertical instead of horizontal. Zimbabwean is played on the light squares.
All checkers variants share the number 23, but the abbreviated version \sphinxcode{\sphinxupquote{{[}GameType "23"{]}}}
still expands to \sphinxcode{\sphinxupquote{{[}GameType "23,B,8,8,N1,0"{]}}} (American pool checkers).

\sphinxAtStartPar
The game types 29, 30, 31, 40 and 41 are not listed in \sphinxcite{introduction:wikipedia}, but they were added
based on conventions of the game site \sphinxhref{http://playok.com}{Play OK} and the
program \sphinxhref{http://aurora.shashki.com/}{Aurora Borealis}.


\begin{savenotes}\sphinxattablestart
\sphinxthistablewithglobalstyle
\centering
\begin{tabular}[t]{*{2}{\X{1}{2}}}
\sphinxtoprule
\sphinxstyletheadfamily 
\sphinxAtStartPar
Attribute
&\sphinxstyletheadfamily 
\sphinxAtStartPar
Description
\\
\sphinxmidrule
\sphinxtableatstartofbodyhook
\sphinxAtStartPar
Start\sphinxhyphen{}colour
&
\sphinxAtStartPar
Either W or B \sphinxhyphen{} white/black side starts
\\
\sphinxhline
\sphinxAtStartPar
Board\sphinxhyphen{}width
&
\sphinxAtStartPar
Width of board.
\\
\sphinxhline
\sphinxAtStartPar
Board\sphinxhyphen{}height
&
\sphinxAtStartPar
Height of board.
\\
\sphinxhline
\sphinxAtStartPar
Notation
&
\sphinxAtStartPar
A character indicating the notation type
\begin{itemize}
\item {} 
\sphinxAtStartPar
A = alpha/numeric like chess,

\item {} 
\sphinxAtStartPar
N = numeric like draughts.

\item {} 
\sphinxAtStartPar
S = SAN \sphinxhyphen{} short\sphinxhyphen{}form notation.

\end{itemize}

\sphinxAtStartPar
followed by a number indicating the location of the
first square (A1 or 1) from the perspective of the
starting player
\begin{itemize}
\item {} 
\sphinxAtStartPar
0 = Bottom left

\item {} 
\sphinxAtStartPar
1 = Bottom right

\item {} 
\sphinxAtStartPar
2 = Top left

\item {} 
\sphinxAtStartPar
3 = Top right

\end{itemize}
\\
\sphinxhline
\sphinxAtStartPar
Invert\sphinxhyphen{}flag
&\begin{itemize}
\item {} 
\sphinxAtStartPar
0 = The bottom left corner is a playing square

\item {} 
\sphinxAtStartPar
1 = The bottom left corner is not a playing square

\end{itemize}
\\
\sphinxbottomrule
\end{tabular}
\sphinxtableafterendhook\par
\sphinxattableend\end{savenotes}

\begin{sphinxadmonition}{important}{Important:}
\sphinxAtStartPar
The interpretation of the Invert\sphinxhyphen{}flag has been changed! The previous
interpretation (0 = Play on dark squares, 1 = Play on light squares)
was not accurate for certain game types.
\end{sphinxadmonition}

\begin{sphinxadmonition}{note}{Note:}
\sphinxAtStartPar
The Start\sphinxhyphen{}colour field is just an indication for the colour of the
pieces, but it has no influence on the notation.
\end{sphinxadmonition}

\begin{sphinxadmonition}{note}{Note:}
\sphinxAtStartPar
The principal direction of the notation is assumed to be horizontal.
This means that square 2 is always to the left or to the right of square 1.
\end{sphinxadmonition}

\begin{sphinxadmonition}{note}{Note:}
\sphinxAtStartPar
Usually the game is played on the dark squares.
\end{sphinxadmonition}


\section{Examples}
\label{\detokenize{gametype:examples}}
\begin{sphinxVerbatim}[commandchars=\\\{\}]
\PYG{p}{[}\PYG{n}{GameType} \PYG{l+s+s2}{\PYGZdq{}}\PYG{l+s+s2}{0}\PYG{l+s+s2}{\PYGZdq{}}\PYG{p}{]}                   \PYG{p}{\PYGZob{}}\PYG{n}{Straight} \PYG{n}{chess}\PYG{p}{\PYGZcb{}}
\PYG{p}{[}\PYG{n}{GameType} \PYG{l+s+s2}{\PYGZdq{}}\PYG{l+s+s2}{0,W,8,8,S0}\PYG{l+s+s2}{\PYGZdq{}}\PYG{p}{]}          \PYG{p}{\PYGZob{}}\PYG{n}{Straight} \PYG{n}{chess} \PYG{k}{with} \PYG{n}{full} \PYG{n}{spec}\PYG{p}{\PYGZcb{}}
\PYG{p}{[}\PYG{n}{GameType} \PYG{l+s+s2}{\PYGZdq{}}\PYG{l+s+s2}{20}\PYG{l+s+s2}{\PYGZdq{}}\PYG{p}{]}                  \PYG{p}{\PYGZob{}}\PYG{l+m+mi}{10}\PYG{n}{x10} \PYG{n}{draughts}\PYG{p}{\PYGZcb{}}
\PYG{p}{[}\PYG{n}{GameType} \PYG{l+s+s2}{\PYGZdq{}}\PYG{l+s+s2}{21,B,8,8,N1,0}\PYG{l+s+s2}{\PYGZdq{}}\PYG{p}{]}       \PYG{p}{\PYGZob{}}\PYG{n}{English} \PYG{n}{draughts} \PYG{k}{with} \PYG{n}{full} \PYG{n}{spec}\PYG{p}{\PYGZcb{}}
\end{sphinxVerbatim}

\sphinxAtStartPar
Italian draughts uses the following notation:
\begin{quote}

\noindent\sphinxincludegraphics{{gametype_italian}.png}
\end{quote}

\sphinxAtStartPar
For Italian draughts the first square (1) is located in the top left corner of
the board. The next square (2) can be found by moving in horizontal direction
to the right. Therefore Italian draughts gets the notation number 2, that
corresponds with (Top left, horizontal).

\sphinxAtStartPar
American pool checkers uses the following notation:
\begin{quote}

\noindent\sphinxincludegraphics{{gametype_american_pool_checkers}.png}
\end{quote}

\sphinxAtStartPar
Pool checkers uses the following notation:
\begin{quote}

\noindent\sphinxincludegraphics{{gametype_unified_pool_checkers}.jpg}
\end{quote}

\sphinxAtStartPar
Jamaican draughts uses the following notation:
\begin{quote}

\noindent\sphinxincludegraphics{{gametype_jamaican_draughts}.png}
\end{quote}

\sphinxstepscope


\chapter{Changelog}
\label{\detokenize{changelog:changelog}}\label{\detokenize{changelog::doc}}
\sphinxAtStartPar
24 June 2017
\begin{itemize}
\item {} 
\sphinxAtStartPar
Put the standard in a github repository on \sphinxurl{https://github.com/wiegerw/pdn}.

\end{itemize}

\sphinxAtStartPar
9 June 2017
\begin{itemize}
\item {} 
\sphinxAtStartPar
Added some time control tag examples provided by Igor Le Masson.

\end{itemize}

\sphinxAtStartPar
1 January 2016
\begin{itemize}
\item {} 
\sphinxAtStartPar
Made some corrections and improvements to the FEN grammar, noted by Rein Halbersma.

\item {} 
\sphinxAtStartPar
Added an archived copy of the PDN 2.0 standard, since it is no longer available online.

\item {} 
\sphinxAtStartPar
Added Zimbabwean pool checkers.

\end{itemize}

\sphinxAtStartPar
3 May 2014
\begin{itemize}
\item {} 
\sphinxAtStartPar
Changed the GameType tag for Jamaican draughts: A5 \sphinxhyphen{}\textgreater{} A1

\item {} 
\sphinxAtStartPar
Removed the vertical direction from the Notation field, since in all
known cases the notation direction is horizontal.

\end{itemize}

\sphinxAtStartPar
30 April 2014
\begin{itemize}
\item {} 
\sphinxAtStartPar
Added a test page for the GameType tag.

\item {} 
\sphinxAtStartPar
Corrected the description of the notation field of the GameType tag.

\item {} 
\sphinxAtStartPar
Made some corrections to the GameType table.
\begin{itemize}
\item {} 
\sphinxAtStartPar
Pool checkers (unified): A1 \sphinxhyphen{}\textgreater{} A0

\item {} 
\sphinxAtStartPar
Turkish draughts: A1 \sphinxhyphen{}\textgreater{} A0

\end{itemize}

\end{itemize}

\sphinxAtStartPar
27 April 2014
\begin{itemize}
\item {} 
\sphinxAtStartPar
Changed the result type of Frisian draughts to Default.

\end{itemize}

\sphinxAtStartPar
18 April 2014
\begin{itemize}
\item {} 
\sphinxAtStartPar
Made some corrections to the gametype table, noted by Rein Halbersma.
\begin{itemize}
\item {} 
\sphinxAtStartPar
Russian/Brazilian/Czech: A1 \sphinxhyphen{}\textgreater{} A0

\item {} 
\sphinxAtStartPar
Spantsiretti: N2 \sphinxhyphen{}\textgreater{} A0

\item {} 
\sphinxAtStartPar
Thai: black starts, and N1 \sphinxhyphen{}\textgreater{} N2

\end{itemize}

\item {} 
\sphinxAtStartPar
Added an improved description of move disambiguation, supplied by Ed Gilbert.

\item {} 
\sphinxAtStartPar
Added a section with proposals for the next version of the standard.

\end{itemize}

\sphinxAtStartPar
27 May 2013
\begin{itemize}
\item {} 
\sphinxAtStartPar
Made some improvements to the gametype section, with the help of Jake Cacher.
\begin{itemize}
\item {} 
\sphinxAtStartPar
Added game types Jamaican draughts and Pool checkers

\item {} 
\sphinxAtStartPar
Added some examples of the board layout of game types

\item {} 
\sphinxAtStartPar
Adapted the interpretation of the invert flag in the game type

\item {} 
\sphinxAtStartPar
Extended the notation flag of the game type, such that the Jamaican draughts notation is supported

\end{itemize}

\item {} 
\sphinxAtStartPar
Added an example for setup commands, supplied by Gérard Taille.

\item {} 
\sphinxAtStartPar
Added a proposal for the notation null moves, by Gérard Taille.

\end{itemize}

\sphinxAtStartPar
12 March 2012
\begin{itemize}
\item {} 
\sphinxAtStartPar
Added a section about character encodings.

\item {} 
\sphinxAtStartPar
Adapted the allowed values for the \sphinxcode{\sphinxupquote{Result}} tag, and added a \sphinxcode{\sphinxupquote{ResultFormat}} tag that can be used to specify uncommon result values.

\item {} 
\sphinxAtStartPar
Added \sphinxcode{\sphinxupquote{clk}}, \sphinxcode{\sphinxupquote{mct}}, \sphinxcode{\sphinxupquote{egt}} and \sphinxcode{\sphinxupquote{emt}} embedded commands.

\item {} 
\sphinxAtStartPar
Added \sphinxcode{\sphinxupquote{TimeControlWhite}} and \sphinxcode{\sphinxupquote{TimeControlBlack}} tags for specifying individual time settings.

\item {} 
\sphinxAtStartPar
Added requirements for notation type and capture separators.

\item {} 
\sphinxAtStartPar
Adapted the grammars to allow alpha numeric moves without separator (a3b4).

\item {} 
\sphinxAtStartPar
Added documentation about when the full capture notation may/must be used.

\item {} 
\sphinxAtStartPar
Added requirement about disambiguation of ambiguous moves.

\item {} 
\sphinxAtStartPar
Added a comment that the empty string is always allowed for a PDN tag value.

\item {} 
\sphinxAtStartPar
Added line comments starting with a \sphinxcode{\sphinxupquote{\%}}.

\item {} 
\sphinxAtStartPar
Added FEN values with unknown color (specified using \sphinxcode{\sphinxupquote{?}}).

\item {} 
\sphinxAtStartPar
Updated the grammars.

\item {} 
\sphinxAtStartPar
Updated the PDN checker.

\end{itemize}

\sphinxAtStartPar
\sphinxstylestrong{Author:} Wieger Wesselink, wieger \textless{}at\textgreater{} 10x10 \textless{}dot\textgreater{} org

\begin{sphinxthebibliography}{Grammati}
\bibitem[Wikipedia]{introduction:wikipedia}
\sphinxAtStartPar
Wikipedia: Portable Draughts Notation \sphinxurl{https://en.wikipedia.org/wiki/Portable\_Draughts\_Notation}
\bibitem[Nemesis]{introduction:nemesis}
\sphinxAtStartPar
PDN 2.0 Specification by Murray Cash (draft) \sphinxcode{\sphinxupquote{pdn2.txt}} An archived copy of \sphinxurl{http://www.nemesis.info/pdn2.txt}, 17 July 2012.
\bibitem[PGN]{introduction:pgn}
\sphinxAtStartPar
Portable Game Notation Specification and Implementation Guide \sphinxurl{http://www.saremba.de/chessgml/standards/pgn/pgn-complete.htm}
\bibitem[Sage]{introduction:sage}
\sphinxAtStartPar
Adrian Millet’s PDN description: \sphinxurl{https://web.archive.org/web/20111021120519/http://homepages.tcp.co.uk/~pcsol/sagehlp1.htm\#PDN}
\bibitem[Grimminck]{introduction:grimminck}
\sphinxAtStartPar
Michel Grimminck’s PDN page \sphinxurl{https://www.xs4all.nl/~mdgsoft/draughts/pdn.html}
\bibitem[DGT]{introduction:dgt}
\sphinxAtStartPar
DGT Clock times extension \sphinxurl{https://web.archive.org/web/20100502070409/http://digitalgametechnology.com/site/index.php/Board-Driver/pgn-clock-times-extension.html}
\bibitem[Forum]{introduction:forum}
\sphinxAtStartPar
PDN standard topic on World Draughts Forum \sphinxurl{https://damforum.nl/bb3/viewtopic.php?t=2544}
\bibitem[DParser]{implementation:id4}
\sphinxAtStartPar
DParser, a GLR parser generator written in C \sphinxurl{http://dparser.sourceforge.net/}
\bibitem[Grammatica]{implementation:id5}
\sphinxAtStartPar
Grammatica, an LL parser generator written in java \sphinxurl{http://grammatica.percederberg.net/}
\bibitem[TPG]{implementation:tpg}
\sphinxAtStartPar
Toy Parser Generator, a parser written in python \sphinxurl{http://cdsoft.fr/tpg/}
\end{sphinxthebibliography}



\renewcommand{\indexname}{Index}
\printindex
\end{document}